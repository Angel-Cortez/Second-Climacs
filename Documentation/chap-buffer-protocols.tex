\chapter{Buffer protocols}
\label{chap-buffer-protocols}

The buffer protocols have been chosen so that they can fit a variety
of editors.  As a consequence, they are not particularly
Emacs-centric.  For example, in Emacs, a newline character is just
another character, so that moving past it using the
\texttt{forward-char} command changes the line in which \emph{point}
is located, and using the \texttt{delete-char} command when
\emph{point} is to the left of a newline character joins the line to
the next one.

In contrast, the buffer protocols documented here are line oriented
and there is no newline character; only a sequence of lines.  At some
level, it is of course desired to have Emacs-compatible commands, but
these commands are written separately, using this buffer protocol to
accomplish the effects.  For example, the Emacs-compatible
\texttt{forward-item} command checks whether it is at the end of a
line, and if so, detaches the cursor from that line and attaches it to
the next one.  Similarly, the Emacs compatible \texttt{delete-item}
command calls \texttt{join-line} in the buffer protocol to obtain the
desired effect when it is at the end of a line.

By writing the editor commands in two levels like this, we hope it
will be easier to use the buffer protocols to write emulators for
other editors, such as \texttt{VIM}.

The buffer participates in two different buffer protocols:

\begin{enumerate}
\item The \emph{edit protocol}, used by client editing and
  cursor-motion operations.
\item The \emph{update protocol}, used by redisplay operations to
  determine what items are contained in the buffer. 
\end{enumerate}

The operations in the edit protocol were designed to be \emph{fast}
(typically around 10 $\mu s$) so that it is practical to use these
operations in a loop, say to insert or delete a region, or to
accomplish several operations inside a keyboard macro.  The exceptions
are the operations \texttt{split-line} and \texttt{join-line} that
take time proportional to the number of items in second line.%
\footnote{We may improve on this performance in the future.}

The operations in the update protocol were designed to be called at
the frequency of the \emph{event loop} of an application, typically
once for each character typed, but also when a window is resized or
scrolled (in which case, these operations are very fast since no
modifications to the buffer have occurred). 

The buffer protocols expose two levels of abstraction to client code: 

\begin{enumerate}
\item The \emph{buffer} level represents the \emph{sequence of lines}
  independently of how the individual lines are represented.  
\item The \emph{line} level represents individual lines. 
\end{enumerate}

As mentioned above, the buffer protocols do not pretend to manage any
equivalence between line breaks and some sequence of characters.  It
is up to client code to model such an equivalence if desired.  As a
consequence, the buffer protocols do not allow for a cursor at the
beginning of a line to more backward or a cursor at the end of a line
to move forward.  An attempt at doing so will result in an error being
signaled.  If client code wants to impose a model where the line break
corresponds to (say) the \emph{newline} character, then it must
explicitly detach and reattach the cursor to a different line in these
cases.  It can manage that in two different ways: either by explicitly
testing for \texttt{beginning-of-line} or \texttt{end-of-line} before
calling the equivalent buffer function, or by handling the error that
results from the attempt.

The buffer also does not interpret the meaning of any of items
contained in it.  For instance, whether an item is to be considered
part of a \emph{word} or not, is not decided at the buffer level, but
at the level of the syntax.  As a consequence, the buffer protocol
does not offer any functions that require such interpretation, such as
\texttt{forward-word}, \texttt{end-of-paragraph}, etc. 

\section{Buffer edit protocol}

\Defgeneric {beginning-of-buffer-p} {cursor}

Return \textit{true} if and only if \textit{cursor} is located at the
beginning of a buffer.

\Defgeneric {end-of-buffer-p} {cursor}

Return \textit{true} if and only if \textit{cursor} is located at the
end of a buffer.

\Defgeneric {beginning-of-line-p} {cursor}

Return \textit{true} if and only if \textit{cursor} is located at the
beginning of a line.

\Defgeneric {end-of-line-p} {cursor}

Return \textit{true} if and only if \textit{cursor} is located at the
end of a line.

\Defgeneric {beginning-of-buffer} {cursor}

Position \textit{cursor} at the very beginning of the buffer.

\Defgeneric {end-of-buffer} {cursor}

Position \textit{cursor} at the very end of the buffer.

\Defgeneric {beginning-of-line} {cursor}

Position \textit{cursor} at the very beginning of the line.

\Defgeneric {end-of-line} {cursor}

Position \textit{cursor} at the very end of the line.

\Defgeneric {forward-item} {cursor}

Move \textit{cursor} forward one position.  If \emph{cursor} is at the
end of the line, the error condition \texttt{end-of-line} will be
signaled.

\Defgeneric {backward-item} {cursor}

Move \textit{cursor} backward one position.  If \emph{cursor} is at
the beginning of the line, the error condition
\texttt{beginning-of-line} will be signaled.

\Defgeneric {insert-item} {cursor item}

Insert an item at the location of \textit{cursor}.  After this
operation, any left-sticky cursor located at the same position as
\textit{cursor} will be located before \textit{item}, and any
right-sticky cursor located at the same position as \textit{cursor}
will be located after \textit{item}.

\Defgeneric {delete-item} {cursor}

Delete the item immediately to the right of \emph{cursor}.  If
\emph{cursor} is at the end of the line, the error condition
\texttt{end-of-line} will be signaled.

\Defgeneric {erase-item} {cursor}

Delete the item immediately to the left of \emph{cursor}.  If
\emph{cursor} is at the beginning of the line, the error condition
\texttt{beginning-of-line} will be signaled.

\Defgeneric {item-after-cursor} {cursor}

Return the item located immediately to the right of \textit{cursor}.
If \emph{cursor} is at the end of the line, the error condition
\texttt{end-of-line} will be signaled.

\Defgeneric {item-before-cursor} {cursor}

Return the item located immediately to the left of \textit{cursor}.
If \emph{cursor} is at the beginning of the line, the error condition
\texttt{beginning-of-line} will be signaled.

\Defgeneric {split-line} {cursor}

Split the line in which \textit{cursor} is located into two lines, the
first cone containing the text preceding \textit{cursor} and the
second one containing the text following \textit{cursor}.  After this
operation, any left-sticky cursor located at the same position as
\textit{cursor} will be located at the end of the first line, and any
right-sticky cursor located at the same position as \textit{cursor}
will be located at the beginning of the second line.

\Defgeneric {join-line} {cursor}

Join the line in which \textit{cursor} is located and the following
line.  If \textit{cursor} is located at the last line of the buffer,
the error condition \texttt{end-of-buffer} will be signaled.

\Defgeneric {detach-cursor} {cursor}

The class of \textit{cursor} is changed to \texttt{detached-cursor}
and it is removed from the line it was initially located in. 

If \textit{cursor} is already detached, this operation has no effect.

\Defgeneric {attach-cursor} {cursor line \optional (position 0)}

Attach \textit{cursor} to \textit{line} at \textit{position}.  If
\textit{position} is supplied and it is greater than the number of
items in \textit{line}, the error condition \texttt{end-of-line} is
signaled.  If \textit{cursor} is already attached to a line, the error
condition \texttt{cursor-attached} will be signaled.

\Defgeneric {cursor-position} {cursor}

Return the position of \textit{cursor} as two values: the line number
and the item number within the line. 

\Defgeneric {line-count} {cursor}

Return the number of lines in the buffer in which \textit{cursor} is
located.

\Defgeneric {item-count} {cursor}

Return the number of items in the line in which \textit{cursor} is
located.

\Defgeneric {line} {cursor}

Return the line in which \textit{cursor} is located. 

\Defgeneric {line-number} {line}

Return the line number of \textit{line}.  The first line of the buffer
has the number $0$. 

\Defgeneric {find-line} {buffer line-number}

Return the line in the buffer with the given \textit{line-number}.  If
\textit{line-number} is less than $0$ then the error
\texttt{beginning-of-buffer} is signaled.  If \textit{line-number} is
greater than or equal to the number of lines in the buffer, then the
error \texttt{end-of-buffer} is signaled.

\Defgeneric {buffer} {object}

Return the buffer in which \textit{object} is located, where
\textit{object} is a \emph{cursor} or a \emph{line}.

Notice that the edit protocol does not contain any
\texttt{delete-line} operation.  This design decision was made on
purpose.  By only providing \texttt{join-line}, we guarantee that
removing a line leaves a \emph{trace} in the buffer in the form of a
modification operation on the first of the two lines that were
joined.  This features is essential in order for the \emph{update
  protocol} to work correctly.

\section{Buffer update protocol}

At the center of the update protocol is the concept of a \emph{time
  stamp}.  Each buffer manages a \emph{current time} which is a
non-negative integer that has the initial value $0$.  Whenever a line
is modified or created, it is marked with the current time of the
buffer, and then the current time of the buffer is incremented.

\Defgeneric {current-time} {buffer}

Return the current time of the buffer, and then increment it. 

\Defgeneric {items} {line \key start end}

Return the items of \textit{line} as a vector.  The keyword parameters
\textit{start} and \textit{end} have the same interpretation as for
\cl{} sequence functions.

\Defgeneric {update} {buffer time sync skip update create}

The \textit{buffer} parameter is a buffer that might have been
modified since the last update operation.  The \textit{time} parameter
is the last time the update operation was called, so that the
\texttt{update} function will report modifications since that time. 

The parameters \textit{sync}, \textit{skip}, \textit{update}, and
\textit{create}, are functions that are called by the \texttt{update}
function.  They are to be considered as \emph{edit operations} on some
representation of the buffer as it was after the previous call to
\texttt{update}.  The operations have the following meaning:

\begin{itemize}
\item \textit{create} indicates a line that has been created.  The
  function is called with that line as an argument.
\item \textit{update}%
  \fixme{Maybe rename this operation \textit{modify}?}
  indicates a line that has been modified.  The
  function is called with that line as an argument.
\item \textit{sync} indicates the first unmodified line after a
  sequence of new or modified lines.  Accordingly, this function is
  called once, following one or more calls to \textit{create} or
  \textit{update}.  This function is called with a single argument:
  the unmodified line.
\item \textit{skip} indicates that a number of lines have not been
  subject to any modifications since the last update.  The function
  takes a single argument, the number of lines to skip.  This function
  is called \emph{first} to indicate that a \emph{prefix} of the
  buffer is unmodified, or after a \emph{sync} operation to indicate
  that that many lines following the one given as argument to the
  \textit{sync} operation are unmodified.%
  \fixme{Decide whether this operation should be called on an
    unmodified \emph{suffix} of the buffer, or whether the last
    \textit{sync} operation is enough.}
\end{itemize}

