\chapter{Representation of the editor buffer}
\label{chap-internals-buffer}

\section{Standard buffer representation}

The standard buffer representation organizes the lines in a
\emph{splay tree} \cite{Sleator:1985:SBS:3828.3835}.  This
organization has several advantages:

\begin{itemize}
\item A line that is modified moves to the root of the tree, and
  recently used lines stay close to the root, making some editing
  operations more efficient.
\item It is computationally cheap to know the line number of the
  current line at all times. 
\end{itemize}

The standard buffer representation uses two different representations
of a \emph{line}.  Whenever a line is modified, it must be
\emph{open}.  In the standard buffer representation, an open line is
represented as a \emph{splay tree} \cite{Sleator:1985:SBS:3828.3835}
of \emph{items}.  Items are typically characters, but any object is
allowed.

