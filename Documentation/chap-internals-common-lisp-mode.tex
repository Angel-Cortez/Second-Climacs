\chapter{\cl{} mode}
\label{chap-internals-common-lisp-mode}

\section{Syntax}

The \cl{} syntax maintains a \emph{splay tree}
\cite{Sleator:1985:SBS:3828.3835} of \emph{paragraphs}.  A paragraph
is a sequence of adjacent lines. There are three kinds of paragraphs:

\begin{itemize}
\item Paragraphs that contain only lines that are either empty or
  that contain only whitespace.  Such a paragraph is called a
  \emph{blank paragraph}. 
\item Paragraphs that contain only lines that start with a semicolon.
  Such a paragraph is called a \emph{comment paragraph}.
\item Paragraphs that contain other lines.  Such a paragraph is called
  a \emph{code paragraph}.
\end{itemize}

\cl{} syntax contains a special version of the \cl{} \emph{reader}.
It differs from the standard reader in the following ways:

\begin{itemize}
\item It never signals an error.
\item It records the start and end position of every call, as well as
  the object read.
\item Instead of calling \texttt{intern} on symbols, it merely records
  that character sequence as being a symbol in the current package. 
\end{itemize}

On the other hand, it behaves like the ordinary \cl{} reader in that
it can handle custom reader macros, even though it provides reader
macros for standard macro characters that behave slightly differently
from the standard reader macros. 

When some illegal syntax is encountered, it tries to do something
reasonable.  For instance if \emph{end of file} is encountered in the
middle of reading a list, the end of file is treated as terminating
the list.  When an illegal token is encountered, an object is returned
that indicates this fact. 


