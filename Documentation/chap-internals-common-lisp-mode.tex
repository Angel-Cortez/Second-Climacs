\chapter{\commonlisp{} mode}

\section{Syntax}

\subsection{Parsing using the \commonlisp{} reader}

We use a special version of the \commonlisp{} reader to parse the
contents of a buffer.  We use a special version of the reader for the
following reasons:

\begin{itemize}
\item We need a different action from that of the standard reader when
  it comes to interpreting tokens.  In particular, we do not
  necessarily want the incremental parser to intern symbols
  automatically, and we do not want the reader to fail when a symbol
  with an explicit package prefix does not exist.
\item We need for the reader to return not only a resulting
  expression, but an object that describes the start and end positions
  in the buffer where the expression was read.
\item The reader needs to return source elements that are not returned
  by an ordinary reader, such as comments and expressions that are
  skipped by certain other reader macros.
\item The reader can not fail when an incorrect character is
  encountered, nor when end of file is encountered in the middle of a
  call.
\end{itemize}

We call the data structure referred to in the last item a \emph{parse
  result}.  It contains the following slots:

\begin{itemize}
\item The start and the end location of the parse result in the
  buffer.  For details on how this parse result is represented,
  \refSec{sec-common-lisp-mode-syntax-data-structure}.
\item The expression that was read, with some possible modifications.
  Tokens are not represented as themselves for reasons mentioned
  above.
\item A list of \emph{children}.  These are parse results that were
  returned by recursive calls to the reader.  The children are
  represented in the order they were encountered in the buffer.  This
  order may be different from the order in which the corresponding
  expressions appear in the expression resulting from the call to the
  reader.
\end{itemize}

\subsection{Data structure}
\label{sec-common-lisp-mode-syntax-data-structure}

The \commonlisp{} syntax maintains a sequence%
\footnote{It is not a \commonlisp{} sequence, but just a suite
  represented in a different way.}
of top-level parse results.
