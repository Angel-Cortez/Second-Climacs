\chapter{\commonlisp{} mode}

\section{Syntax}

\subsection{Parsing using the \commonlisp{} reader}

We use a special version of the \commonlisp{} reader to parse the
contents of a buffer.  We use a special version of the reader for the
following reasons:

\begin{itemize}
\item We need a different action from that of the standard reader when
  it comes to interpreting tokens.  In particular, we do not
  necessarily want the incremental parser to intern symbols
  automatically, and we do not want the reader to fail when a symbol
  with an explicit package prefix does not exist or when the package
  corresponding to the package prefix does not exists.
\item We need for the reader to return not only a resulting
  expression, but an object that describes the start and end positions
  in the buffer where the expression was read.
\item The reader needs to return source elements that are not returned
  by an ordinary reader, such as comments and expressions that are
  skipped by certain other reader macros.
\item The reader can not fail when an incorrect character is
  encountered, nor when end of file is encountered in the middle of a
  call.
\end{itemize}

We call the data structure referred to in the last item a \emph{parse
  result}.  It contains the following slots:

\begin{itemize}
\item The start and the end location of the parse result in the
  buffer.  For details on how this parse result is represented,
  \refSec{sec-common-lisp-mode-syntax-data-structure}.
\item The expression that was read, with some possible modifications.
  Tokens are not represented as themselves for reasons mentioned
  above.
\item A list of \emph{children}.  These are parse results that were
  returned by recursive calls to the reader.  The children are
  represented in the order they were encountered in the buffer.  This
  order may be different from the order in which the corresponding
  expressions appear in the expression resulting from the call to the
  reader.
\end{itemize}

\subsection{Data structure}
\label{sec-common-lisp-mode-syntax-data-structure}

A location in the buffer is considered a \emph{top-level location} if
and only if, when the buffer is parsed by a number of consecutive
calls to \texttt{read}, when this location is reached, the reader is
in its initial state with no recursive pending invocations.

The \commonlisp{} syntax maintains a sequence%
\footnote{It is not a \commonlisp{} sequence, but just a suite
  represented in a different way.}  of \emph{top-level parse results}.
A parse result is considered top-level if it is the result of an
immediate call to \texttt{read}, as opposed to of a recursive call.

This sequence is organized as two ordinary \commonlisp{} lists, called
the \emph{prefix} and the \emph{suffix}.  Given a top-level location
$L$ in the buffer, the prefix contains a list of the top-level parse
results that precede $L$ and the suffix contains a list of the
top-level parse results that follow $L$.  The top-level parse results
in the prefix occur in reverse order compared to order in which they
appear in the buffer.  The top-level parse results in the suffix occur
in the same order as they appear in the buffer.  the location $L$ is
typically immediately before or immediately after the top-level
expression in which the \emph{cursor} of the current view is located,
but that is not a requirement.  \refFig{fig-cl-parser-prefix-suffix}
illustrates the prefix and the suffix of a buffer with five top-level
expressions.

\begin{figure}
\begin{center}
\inputfig{fig-cl-parser-prefix-suffix.pdf_t}
\end{center}
\caption{\label{fig-cl-parser-prefix-suffix}
Prefix and suffix containing top-level parse results.}
\end{figure}

Either the prefix or the suffix or both may be the empty list.  The
location $L$ may be moved.  It suffices%
\footnote{Some slots also need to be updated as will be discussed later.}
to pop an element off of one
of the lists and push it onto the other.

The representation of a parse result is shown in
\refFig{fig-parse-result}.

\begin{figure}
\begin{center}
\inputfig{fig-parse-result.pdf_t}
\end{center}
\caption{\label{fig-parse-result}
Representation of parse result.}
\end{figure}

Let the \emph{initial character} of some parse result be the first
non-whitespace character encountered during the call to the reader
that produced this parse result.  Similarly, let the \emph{final
  character} of some parse result be the last character encountered
during the call to the reader that produced this parse result,
excluding any look-ahead character that could be un-read before the
parse result was returned.

The slot named \texttt{start-line} is computed as follows:

\begin{itemize}
\item If the parse result is one of the top-level parse results in the
  prefix or the first top-level parse result in the suffix, then the
  value of this slot is the absolute line number of the initial
  character of this parse result.  The first line of the buffer is
  numbered $0$.
\item If the parse result is a top-level parse result in the suffix
  other than the first one, then the value of this slot is the number
  of lines between the value of the slot \texttt{start-line} of the
  preceding parse result and the initial character of this parse
  result.  A value of $0$ indicates the same line as the
  \texttt{start-line} of the preceding parse result.
\item If this parse result is the first in a list of children of some
  parent parse result, then the value of this slot is the number of
  lines between the value of the slot \texttt{start-line} of the parent
  parse result and the initial character of this parse result.
\item If this parse result is the child other than the first in a list
  of children of some parent parse result, then the value of this slot
  is the number of lines between the value of the slot
  \texttt{start-line} of the preceding sibling parse result and the
  initial character of this parse result.
\end{itemize}

The value of the slot \texttt{start-column} is the absolute column
number of the initial character of this parse result.  A value of $0$
means the leftmost column.

The value of the slot \texttt{end-line} of some parse result is the
number of lines between the value of the slot \texttt{start-line} of
the same parse result and the final character of the parse result.
If the parse result starts and ends on the same line, then the value
of this slot is $0$.

The value of the slot \texttt{end-column} is the absolute column
number of the final character of the parse result.

To illustrate the data structure, we use the following example:

\begin{verbatim}
...
34 (f 10)
35
36 (let ((x 1)
37       (y 2))
38   (g (h x)
39      (i y)
40      (j x y)))
41
42 (f 20)
...
\end{verbatim}

Each line is preceded by the absolute line number.  If the parse
result starting at line 36 is the prefix or if it is the first element
of the suffix, it would be represented like this:

\begin{verbatim}
36 04 (let ((x 1) (y 2)) (g (h x) (i y) (j x y)))
   00 01 ((x 1) (y 2))
      00 00 (x 1)
      01 00 (y 2)
   02 02 (g (h x) (i y) (j x y))
      00 00 (h x)
      01 00 (i y)
      02 00 (j x y)
\end{verbatim}

Since column numbers are uninteresting for our illustration, we
show only line numbers.  Furthermore, we present a list as a table for
a more compact presentation.

\subsection{Moving top-level parse results}

Occasionally, some top-level parse results need to be moved from the
prefix to the suffix or from the suffix to the prefix.  There could be
several reasons for such moves:

\begin{itemize}
\item The place between the prefix and the suffix must always be near
  the part of the buffer currently on display when the contents are
  presented to the user.  If the part on display changes as a result
  of scrolling or as a result of the user moving the current cursor,
  then the prefix and suffix must be adjusted to reflect the new
  position prior to the presentation.
\item After items have been inserted into or deleted from the buffer,
  the incremental parser may have to adjust the prefix and the suffix
  so that the altered top-level parse results are near the beginning
  of the suffix.
\end{itemize}

These adjustments are always accomplished by repeatedly moving a
single top-level parse result.

To move a single top-level parse result $P$ from the prefix to the
suffix, the following actions are executed:

\begin{enumerate}
\item Modify the slot \texttt{start-line} of the first parse result of
  the suffix so that, instead of containing the absolute line number,
  it contains the line number relative the value of the slot
  \texttt{start-line} of $P$.
\item Pop $P$ from the prefix and push it onto the suffix.  Rather
  than using the straightforward technique, the \texttt{cons} cell
  referring to $P$ can be reused so as to avoid unnecessary consing.
\end{enumerate}

To move a single top-level parse-result $P$ from the suffix to the
prefix, the following actions are executed:

\begin{enumerate}
\item If $P$ has a successor $S$ in the suffix, then the slot
  \texttt{start-line} of $S$ is adjusted so that it contains the
  absolute line number as opposed to the line number relative to the
  slot \texttt{start-line} of $P$.
\item Pop $P$ from the suffix and push it onto the prefix.  Rather
  than using the straightforward technique, the \texttt{cons} cell
  referring to $P$ can be reused so as to avoid unnecessary consing.
\end{enumerate}

We illustrate this process by showing four possible top-level
locations in the example buffer.  If all three top-level parse results
are located in the suffix, we have the following situation:

\begin{verbatim}
prefix
...
suffix
34 00 (f 10)
02 04 (let ((x 1) (y 2)) (g (h x) (i y) (j x y)))
06 00 (f 20)
...
\end{verbatim}

In the example, we do not show the children of the top-level parse
results.

If the prefix contains the first top-level expression and the suffix
the other two, we have the following situation:

\begin{verbatim}
prefix
...
34 00 (f 10)
suffix
36 04 (let ((x 1) (y 2)) (g (h x) (i y) (j x y)))
06 00 (f 20)
...
\end{verbatim}

If the prefix contains the first two top-level expressions and the suffix
the remaining one, we have the following situation:

\begin{verbatim}
prefix
...
34 00 (f 10)
36 04 (let ((x 1) (y 2)) (g (h x) (i y) (j x y)))
suffix
42 00 (f 20)
...
\end{verbatim}

Finally, if the prefix contains all three top-level expressions, we
have the following situation:


\begin{verbatim}
prefix
...
34 00 (f 10)
36 04 (let ((x 1) (y 2)) (g (h x) (i y) (j x y)))
42 00 (f 20)
suffix
...
\end{verbatim}

\subsection{Incremental update}

Modifications to the buffer are reported at the granularity of entire
lines.  The following operations are possible:

\begin{itemize}
\item A line may be modified.
\item A line may be inserted.
\item A line may be deleted.
\end{itemize}

Several different lines may be modified between two incremental
updates, and in different ways.  The first step in an incremental
update step is to invalidate parse results that are no longer known to
be correct after these modifications.  This step modifies the data
structure described in
\refSec{sec-common-lisp-mode-syntax-data-structure} in the following
way:

\begin{itemize}
\item After the invalidation step, the prefix contains the parse
  results preceding the first modified line, so that these parse
  results are still valid.
\item The suffix contains those parse results following the last
  modified line.  These parse results are still valid, but they may no
  longer be top-level parse results, because the nesting may have
  changed as a result of the modifications preceding the suffix.
\item An additional list of \emph{residual parse results} is created.
  This list contains parse results that have not been invalidated by
  the modifications, i.e. that appear only in lines that have not been
  modified.
\end{itemize}

The order of the parse results in the list of residual parse results
is the same as the order of the parse results in the buffer.  The slot
\texttt{start-line} of each parse result in the list is the absolute
line number of the initial character of that parse result.

Suppose, for example, that the buffer contents in our running example
was modified so that line 37 was altered in some way, and a line was
inserted between the lines 39 and 40.  As a result of this update, we
need to represent the following parse results:

\begin{verbatim}
...
34 (f 10)
35
36       (x 1)
37
38      (h x)
39      (i y)
40
41      (j x y)
42
43 (f 20)
...
\end{verbatim}

In other words, we need to obtain the following representation:

\begin{verbatim}
prefix
...
34 00 (f 10)
residual
36 00 (x 1)
38 00 (h x)
39 00 (i y)
41 00 (j x y)
suffix
43 00 (f 20)
...
\end{verbatim}

\subsubsection{Processing modifications}

While the list of residual parse results is being constructed, its
elements are in the reverse order.  Only when all buffer updates have
been processed is the list of residual parse results reversed to
obtain the final representation.

All line modifications are reported in increasing order of line
number.  Before the first modification is processed, the prefix and
the suffix are positioned as indicated above, and the list of residual
parse results is initialized to the empty list.

The following actions are taken, depending on the position of the
modified line with respect to the suffix, and on the nature of the
modification:

\begin{itemize}
\item If a line has been modified, and either the suffix is empty or
  the modified line precedes the first parse result of the suffix,
  then no action is taken.
\item If a line has been deleted, and the suffix is empty, then no
  action is taken.
\item If a line has been deleted, and it precedes the first parse
  result of the suffix, then the slot \texttt{start-line} of the first
  parse result of the suffix is decremented.
\item If a line has been inserted, and the suffix is empty, then no
  action is taken.
\item If a line has been inserted, and it precedes the first parse
  result of the suffix, then the slot \texttt{start-line} of the first
  parse result of the suffix is incremented.
\item If a line has been modified and the entire first parse result of
  the suffix is entirely contained in this line, then remove the first
  parse result from the suffix and start the entire process again with
  the same modified line.  To remove the first parse result from the
  suffix, first adjust the slot \texttt{start-line} of the second
  element of the suffix (if any) to reflect the absolute start line.
  Then pop the first element off the suffix.
\item If a line has been modified, deleted, or inserted, in a way that
  may affect the first parse result of the suffix, then this parse
  result is first removed from the suffix and then processed as
  indicated below.  Finally, start the entire process again with the
  same modified line.  To remove the first parse result from the
  suffix, first adjust the slot \texttt{start-line} of the second
  element of the suffix (if any) to reflect the absolute start line.
  Then pop the first element off the suffix.
\end{itemize}

Let us see how we process the modifications in our running example.

Line 37 has been altered, so our first task is to adjust the prefix
and the suffix so that the prefix contains the last parse result that
is unaffected by the modifications.  This adjustment results in the
following situation:

\begin{verbatim}
prefix
...
34 00 (f 10)
residue
worklist
suffix
36 04 (let ((x 1) (y 2)) (g (h x) (i y) (j x y)))
06 00 (f 20)
...
\end{verbatim}

The first parse result of the suffix is affected by the fact that line
37 has been modified.  We must move the children of that parse result
to the worklist.  In doing so, we make the \texttt{start-line} of the
first child reflect the absolute line number, and we also make the
\texttt{start-line} of the next parse result of the suffix also
reflect the absolute line number.  We obtain the following situation:

\begin{verbatim}
prefix
...
34 00 (f 10)
residue
worklist
36 01 ((x 1) (y 2))
02 02 (g (h x) (i y) (j x y))
suffix
42 00 (f 20)
...
\end{verbatim}

The first element of the worklist is affected by the modification of
line 37.  We therefore remove it from the worklist, and push the list
of its children as a new level on the worklist.  In doing so, we make
the \texttt{start-line} of the remaining element of the worklist and
that of the first child to be pushed reflect absolute line numbers.
We obtain the following situation:

\begin{verbatim}
prefix
...
34 00 (f 10)
residue
worklist
36 00 (x 1)
01 00 (y 2)
--
38 02 (g (h x) (i y) (j x y))
suffix
42 00 (f 20)
...
\end{verbatim}

The first element of the worklist is unaffected by the modification,
because it precedes the modified line entirely.  We therefore move it
to the residue list.  In doing so, we make the \texttt{start-line}
slot of the next element on the top of the worklist reflect the
absolute line number.  We now have the following situation:

\begin{verbatim}
prefix
...
34 00 (f 10)
residue
36 00 (x 1)
worklist
37 00 (y 2)
--
38 02 (g (h x) (i y) (j x y))
suffix
42 00 (f 20)
...
\end{verbatim}

The first parse result of the top of the worklist is affected by the
modification.  It has no children, so we pop it off the worklist.
Furthermore, there are no more parse result left on the top of the
worklist, so we remove the entire top of the worklist.

\begin{verbatim}
prefix
...
34 00 (f 10)
residue
36 00 (x 1)
worklist
38 02 (g (h x) (i y) (j x y))
suffix
42 00 (f 20)
...
\end{verbatim}

The modification of line 37 is now entirely processed.  We know this
because the first parse result on the worklist occurs beyond the
modified line in the buffer.  We therefore start processing the line
inserted between the existing lines 39 and 40.  The first item on the
worklist is affected by this insertion.  We therefore remove it from
the worklist and push its children instead.  In doing so, we make the
\texttt{start-line} slot of the first child reflect the absolute line
number.  We obtain the following result:

\begin{verbatim}
prefix
...
34 00 (f 10)
residue
36 00 (x 1)
worklist
38 00 (h x)
01 00 (i y)
02 00 (j x y)
suffix
42 00 (f 20)
...
\end{verbatim}

The first element of the worklist is unaffected by the insertion
because it precedes the inserted line entirely.  We therefore move it
to the residue list.  In doing so, we make the \texttt{start-line}
slot of the next element on the top of the worklist reflect the
absolute line number.  We now have the following situation:

\begin{verbatim}
prefix
...
34 00 (f 10)
residue
36 00 (x 1)
38 00 (h x)
worklist
39 00 (i y)
02 00 (j x y)
suffix
42 00 (f 20)
...
\end{verbatim}

Once again, the first element of the worklist is unaffected by the
insertion because it precedes the inserted line entirely.  We
therefore move it to the residue list.  In doing so, we make the
\texttt{start-line} slot of the next element on the top of the
worklist reflect the absolute line number.  We now have the following
situation:

\begin{verbatim}
prefix
...
34 00 (f 10)
residue
36 00 (x 1)
38 00 (h x)
39 00 (i y)
worklist
40 00 (j x y)
suffix
42 00 (f 20)
...
\end{verbatim}

The first element of the worklist is affected by the insertion, in
that it must have its line number incremented.  In fact, the first
element of each level of the worklist and also the first element of
the suffix must have their line numbers incremented.  Furthermore,
this update finishes the processing of the inserted line.  We
now have the following situation:

\begin{verbatim}
prefix
...
34 00 (f 10)
residue
36 00 (x 1)
38 00 (h x)
39 00 (i y)
worklist
41 00 (j x y)
suffix
43 00 (f 20)
...
\end{verbatim}

With no more buffer modifications to process, we terminate the
procedure by moving remaining parse results from the worklist to the
residue list.  Every time a parse result is moved, we make the
\texttt{start-line} slot of the next element reflect the absolute line
number as before.  The final situation is shown here:

\begin{verbatim}
prefix
...
34 00 (f 10)
residue
36 00 (x 1)
38 00 (h x)
39 00 (i y)
41 00 (j x y)
worklist
suffix
43 00 (f 20)
...
\end{verbatim}
