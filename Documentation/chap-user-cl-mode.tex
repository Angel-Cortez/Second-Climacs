\chapter{\cl{} mode}

The contents of this chapter was largely written long before the \cl{}
mode was implemented or even fully designed.  We still use the present
tense to describe functionality, though, as if it existed. 

\section{Syntax-related information}

Syntax-related information is presented as slight alterations of how
the code elements in the buffer is displayed to the user.  The
following methods are used (roughly in decreasing order of frequency):

\begin{itemize}
\item Changing the background color.
\item Changing the foreground color.
\item Changing the font face.
\item Using a different glyph.
\end{itemize}

In addition, a code element that is marked in this way also has
associated textual information that can be read in the minibuffer when
the cursor is positioned on the text, or as a tooltip when the pointer
is positioned above the text.  In the remainder of this chapter, we
give the English version of the textual information.
Internationalization may change the language according to user
preference.

Names of specific colors that are mentioned in this section are merely
\emph{examples}.  They may or may not be the default colors actually
used, and every color can be customized by the end user. 

Furthermore, in some cases a code element that is marked this way has
a \emph{context menu} associated with it.

\subsection{Information about symbols}

In general, errors and warnings are indicated by a specific
\emph{background color}.  Typically these colors are in shades of
\emph{red}. 

Symbols that have no error or warning information associated with them
are displayed using some dark foreground color so as to give good
contrast with the background.  

Except for symbol with lexical bindings and symbols used for
\texttt{loop} keywords, the \emph{hue} of the color is associated with
the \emph{home package} of the symbol.  We use \emph{blue} for the
\cl{} package, \emph{green} for the current package, and
\emph{magenta} for other packages.

Symbols with lexical bindings use \emph{cyan} colors, independently of
the home package of the symbol.  Symbols used as \texttt{loop}
keywords use a dark brown color. 

Within a family of similar hues, the darkest color is used for symbols
used as \emph{functions} and for symbols used as \emph{global
  variables}  The two categories can be distinguished because of the
\emph{position} within an expression.  Slightly lighter colors are
used for \emph{macros} and \emph{symbol macros}.  

A \emph{context menu} may be associated with certain symbols.

\subsubsection{Invalid symbol syntax}

A token which does not have any interpretation as a number is
considered a potential symbol.  If so, there are a few cases where
token is nevertheless invalid as the name of a symbol.

The token might have an invalid constellation of package markers:

\begin{itemize}
\item Too many package markers.
\item Two package markers that are separated by some other character. 
\item A package marker at the end of the token.
\end{itemize}

In this case, the background color of the package markers is vivid
red, and the background of the symbol itself is pink. 

The associated textual information says ``Illegal constellation of
package markers''. 

\refFig{fig-invalid-package-markers} illustrates how this type of
information is displayed.

\begin{figure}
\begin{center}
\inputfig{fig-invalid-package-markers.pdf_t}
\end{center}
\caption{\label{fig-invalid-package-markers}
Display of potential symbol with illegal package markers.}
\end{figure}

No context menu is suggested. 

\subsubsection{Non-existing package}

A symbol with one or two package markers but where the package
indicated in the prefix does not exists is marked with a pink
background, and the package name is marked with a vivid red
background. 

The associated textual information says ``Non-existing package''.

\refFig{fig-non-existing-package} illustrates how this type of
information is displayed.

\begin{figure}
\begin{center}
\inputfig{fig-non-existing-package.pdf_t}
\end{center}
\caption{\label{fig-non-existing-package}
Display of potential symbol with a non-existing package.}
\end{figure}

The context menu has a single option: ``Create the package''.

\subsubsection{Non-existing symbol}

A symbol token with a single package marker that refers to a
non-existing symbol is marked with a vivid red background.  The
associated textual information says ``Non-existing symbol''.  

\refFig{fig-non-existing-symbol} illustrates how this type of
information is displayed.

\begin{figure}
\begin{center}
\inputfig{fig-non-existing-symbol.pdf_t}
\end{center}
\caption{\label{fig-non-existing-symbol}
Display of a non-existing symbol with one package marker.}
\end{figure}

The context menu gives the following options:

\begin{itemize}
\item Create and export the symbol in the specified package.
\item Import the symbol from a different package into the specified
  package, and also export it from the specified package.  The user
  will be prompted for the package to import from.
\end{itemize}

A symbol token with a two package markers that refers to a
non-existing symbol is marked with a pink background.  The
associated textual information says ``Non-existing symbol''.  

\refFig{fig-non-existing-symbol-two-package-markers} illustrates how
this type of information is displayed.

\begin{figure}
\begin{center}
\inputfig{fig-non-existing-symbol-two-package-markers.pdf_t}
\end{center}
\caption{\label{fig-non-existing-symbol-two-package-markers}
Display of a non-existing symbol with two package markers.}
\end{figure}

The context menu gives the following options:

\begin{itemize}
\item Create the symbol in the specified package.
\item Import the symbol from a different package into the specified
  package.  The user will be prompted for the package to import from.
\end{itemize}

\subsubsection{Unexported symbol}

A symbol token with a single package marker that refers to an
existing, but unexported symbol is marked with a pink background.  The
associated textual information says ``Unexported symbol.

\refFig{fig-unexported-symbol} illustrates how this type of
information is displayed.

\begin{figure}
\begin{center}
\inputfig{fig-unexported-symbol.pdf_t}
\end{center}
\caption{\label{fig-unexported-symbol}
Display of an unexported symbol.}
\end{figure}

The context menu gives the following options:

\begin{itemize}
\item Export the symbol from the specified package.
\end{itemize}

\subsubsection{Illegal use of \cl{} symbols}

When a symbol from the \texttt{common-lisp} package is used in a
context that can be determined illegal, it is signaled by the use of
an orange background.

\refFig{fig-illegal-use-of-cl-symbol} illustrates how this type of
information is displayed.

\begin{figure}
\begin{center}
\inputfig{fig-illegal-use-of-cl-symbol.pdf_t}
\end{center}
\caption{\label{fig-illegal-use-of-cl-symbol}
Display of illegal use of \cl{} symbol.}
\end{figure}

\subsubsection{Names of lexical functions}

Names of functions introduced by \texttt{flet} or \texttt{labels} are
shown with a dark \emph{cyan} foreground color, independently of the
home package of the symbol including when that package happens to be
the \texttt{common-lisp} package.

When the pointer is located over such a name, the corresponding symbol
in the introducing binding is highlighted with a light blue
background. 

One entry in the context menu is to jump to location where the binding
was established. 

\subsubsection{Names of lexical variables}

Names of lexical variables introduced by \texttt{let}, \texttt{let*},
\texttt{multiple-value-bind}, and by macros that expand into one of
these special forms are shown with a dark \emph{cyan}
foreground color, independently of the home package of the symbol
including when that package happens to be the \texttt{common-lisp}
package.

When the pointer is located over such a name, the corresponding symbol
in the introducing binding is highlighted with a light blue
background. 

One entry in the context menu is to jump to location where the binding
was established. 

\subsubsection{Names of local macros}

Names of local macros are shown with a slightly lighter \emph{cyan}
foreground color, independently of the home package of the symbol
including when that package happens to be the \texttt{common-lisp}
package.

When the pointer is located over such a name, the corresponding symbol
in the introducing binding is highlighted with a light blue
background. 

One entry in the context menu is to jump to location where the binding
was established. 

\subsubsection{Names of local symbol macros}

Names of lexical variables are shown with a slightly lighter
\emph{cyan} foreground color, independently of the home package of the
symbol including when that package happens to be the
\texttt{common-lisp} package.

When the pointer is located over such a name, the corresponding symbol
in the introducing binding is highlighted with a light blue
background. 

One entry in the context menu is to jump to location where the binding
was established. 

\subsubsection{Names of global functions in the current package}

A dark \emph{green} foreground color is used to display a symbol in the
current package that is the name of a global function, and that is not
used as lexical name (of a variable, a function, a macro, or a symbol
macro).  

The \emph{tooltip} shows the \texttt{documentation} entry associated
with the named function.

If the source location where the function was defined can be
determined, then one entry in the context menu is to jump to that
location.

\subsubsection{Names of global macros in the current package}

A slightly lighter \emph{green} foreground color is used to display a
symbol in the current package that is the name of a global macro, and
that is not used as lexical name (of a variable, a function, a macro,
or a symbol macro).

The \emph{tooltip} shows the \texttt{documentation} entry associated
with the named macro.

If the source location where the macro was defined can be
determined, then one entry in the context menu is to jump to that
location.

\subsubsection{Names of global symbol macros in the current package}

A slightly lighter \emph{green} foreground color is used to display a
symbol in the current package that is the name of a global symbol
macro, and that is not used as lexical name (of a variable, a
function, a macro, or a symbol macro).

The \emph{tooltip} shows the \texttt{documentation} entry associated
with the named symbol macro.

If the source location where the symbol macro was defined can be
determined, then one entry in the context menu is to jump to that
location.

\subsubsection{Names of special variables in the current package}

A dark \emph{green} foreground color is used to display a symbol in
the current package if it is in a context where it is the name of a
special variable.

The \emph{tooltip} shows the \texttt{documentation} entry associated
with the named variable.

If the symbol names a variable that is globally special, and if
the source location where the variable was defined can be
determined, then one entry in the context menu is to jump to that
location.

\subsubsection{Names of constant variables in the current package}

\subsubsection{Names of global functions in the \texttt{common-lisp} package}

A dark \emph{blue} foreground color is used to display a symbol in the
\texttt{common-lisp} package that is the name of a global function,
and that is not used as lexical name (of a variable or a symbol
macro).

The \emph{tooltip} shows the \texttt{documentation} entry associated
with the named function.

One entry of the context menu is to show the entry in the \hs{}
associated with the named function.

\subsubsection{Names of global macros in the \texttt{common-lisp} package}

A slightly lighter \emph{blue} foreground color is used to display a
symbol in the \texttt{common-lisp} package that is the name of a
global macro, and that is not used as lexical name (of a variable or a
symbol macro).

The \emph{tooltip} shows the \texttt{documentation} entry associated
with the named macro.

One entry of the context menu is to show the entry in the \hs{}
associated with the named macro.

\subsubsection{Names of special variables in the \texttt{common-lisp} package}

A dark \emph{blue} foreground color is used to display a symbol in the
\texttt{common-lisp} package that is the name of a special variable.

The \emph{tooltip} shows the \texttt{documentation} entry associated
with the named variable.

One entry of the context menu is to show the entry in the \hs{}
associated with the named variable.

\subsubsection{Names of constant variables in the \texttt{common-lisp} package}

The \emph{tooltip} shows the \texttt{documentation} entry associated
with the named constant variable.

One entry of the context menu is to show the entry in the \hs{}
associated with the named macro.

\subsubsection{Names of special operators}

The \emph{tooltip} shows the \texttt{documentation} entry associated
with the named special operator.

One entry of the context menu is to show the entry in the \hs{}
associated with the named special operator.

\subsubsection{Names of global functions in other packages}

A dark \emph{magenta} foreground color is used to display a symbol in
a package other than the current one or the \texttt{common-lisp}
package that is the name of a global function, and that is not used as
lexical name (of a variable, a function, a macro, or a symbol macro).

The \emph{tooltip} shows the \texttt{documentation} entry associated
with the named function.

If the source location where the function was defined can be
determined, then one entry in the context menu is to jump to that
location.

\subsubsection{Names of global macros in other packages}

A slightly lighter \emph{magenta} foreground color is used to display
a symbol in a package other than the current one or the
\texttt{common-lisp} package that is the name of a global macro, and
that is not used as lexical name (of a variable, a function, a macro,
or a symbol macro).

The \emph{tooltip} shows the \texttt{documentation} entry associated
with the named macro.

If the source location where the macro was defined can be determined,
then one entry in the context menu is to jump to that location.

\subsubsection{Names of global symbol macros in other packages}

A slightly lighter \emph{magenta} foreground color is used to display
a symbol in a package other than the current one or the
\texttt{common-lisp} package that is the name of a global symbol
macro, and that is not used as lexical name (of a variable, a
function, a macro, or a symbol macro).

The \emph{tooltip} shows the \texttt{documentation} entry associated
with the named symbol macro.

If the source location where the macro was defined can be determined,
then one entry in the context menu is to jump to that location.

\subsubsection{Names of special variables in other packages}

A dark \emph{magenta} foreground color is used to display a symbol in
a package other than the current one or the \texttt{common-lisp}
package if it is in a context where it is the name of a special
variable.

The \emph{tooltip} shows the \texttt{documentation} entry associated
with the named variable.

If the symbol names a variable that is globally special, and if
the source location where the variable was defined can be
determined, then one entry in the context menu is to jump to that
location.

\subsubsection{Names of constant variables in other packages}

\subsection{Extraneous whitespace}

Whitespace is considered \emph{extraneous} in the following cases: 

\begin{itemize}
\item When it follows a left parenthesis.
\item When it precedes a right parenthesis.
\item When it follows other whitespace that separate two expressions. 
\item When it follows the last non-whitespace character of a line.
\end{itemize}

It is \emph{not} considered extraneous in the following cases:

\begin{itemize}
\item When it precedes the first non-whitespace character on a line. 
\item When it separates the first semicolon on a line from the last
  preceding non-whitespace character. 
\end{itemize}

Extraneous whitespace is marked with a \emph{pink} background.  The
tooltip associated with the marked background says ``Extraneous
whitespace''. 

\refFig{fig-extraneous-whitespace} illustrates how this type of
information is displayed.

\begin{figure}
\begin{center}
\inputfig{fig-extraneous-whitespace.pdf_t}
\end{center}
\caption{\label{fig-extraneous-whitespace}
Display of extraneous whitespace.}
\end{figure}

\subsection{Argument mismatch}

Argument mismatch is a situation that may occur for the following
types of forms:

\begin{itemize}
\item Function calls.
\item Macro calls.
\item Special forms.
\end{itemize}

Argument mismatch corresponds to one of the following situations:

\begin{itemize}
\item The number of arguments is less than the minimum number required
  by the function, the macro, or the special operator.
\item The number of arguments is greater than the maximum number
  allowed by the function, the macro, or the special operator.
\item The function, the macro or the special operator admit \&key
  arguments, and there is an odd number of arguments in that part of
  the argument list.
\item The function, the macro or the special operator requires the
  number of arguments to be even, but an odd number of arguments are
  given (e.g., for \texttt{setq} and \texttt{setf}).
\item The function, the macro or the special operator requires the
  number of arguments to be odd, but an even number of arguments are
  given.
\end{itemize}

In the case where the number of arguments is greater than the maximum
number allowed, each extraneous arguments is highlighted with a vivid
\emph{red} background.  The tooltip associated with each arguments
says ``Extraneous argument.'' 

In all other cases, the closing parenthesis is preceded by a
\emph{red} rectangle.  Notice that this red rectangle does not
constitute an item in the buffer, and as a consequence it is
impossible to position the cursor between the red rectangle and the
closing parenthesis.  Rather, the red rectangle is part of the way the
closing parenthesis is displayed.  In the case where too few arguments
were given, the tooltip says ``Too few arguments'', and in the other
cases it says ``Wrong argument parity''.

\subsection{Indentation}

A line that is indented incorrectly is displayed with a \emph{pink}
arrow in the left margin.  If the indentation of the line is
insufficient, the arrow points to the right, and if the indentation of
the line is excessive, then the arrow points to the left.

\refFig{fig-wrong-indentation} illustrates how this type of
information is displayed.

\begin{figure}
\begin{center}
\inputfig{fig-wrong-indentation.pdf_t}
\end{center}
\caption{\label{fig-wrong-indentation}
Display of extraneous whitespace.}
\end{figure}

\subsection{Comments}

%%  LocalWords:  tooltip minibuffer unexported whitespace
