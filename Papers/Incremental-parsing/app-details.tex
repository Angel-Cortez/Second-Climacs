\section{Cache representation}
\label{app-cache-representation}

The sequence of top-level parse results is split into a \emph{prefix}
and a \emph{suffix}, typically reflecting the current position in the
buffer being edited by the end user.  The suffix contains parse
results in the order they appear in the buffer, whereas the prefix
contains parse results in reverse order, making it easy to move parse
results between the prefix and the suffix.

Depending on the location of the parse result in the cache data
structure, its position may be \emph{absolute} or \emph{relative}.
The prefix contains parse results that precede updates to the buffer.
For that reason, these parse results have absolute positions.  Parse
results in the suffix, on the other hand, follow updates to the
buffer.  In particular, if a line is inserted or deleted, the parse
results in the suffix will have their positions changed.  For that
reason, only the first parse result of the suffix has an absolute
position.  Each of the others has a position relative to its
predecessor.  When a line is inserted or deleted, only the first parse
result of the suffix has to have its position updated.  When a parse
result is moved from the prefix to the suffix, or from the suffix to
the prefix, the positions concerned are updated to maintain this
invariant.

To avoid having to traverse all the descendants of a parse result when
its position changes, we make the position of the first child of some
parse result $P$ relative to that of $P$, and the children, other than
the first, of some parse result $P$, have positions relative to the
previous child in the list.

\subsection{Invalidation phase}

\subsubsection{Description}

During the invalidation phase, two additional lists of parse results
are maintained, called the \emph{worklist} and the \emph{residue}.  At
the beginning of this phase, these lists are both empty.  At the end
of this phase, any remaining parse results on the worklist are moved
to the residue, resulting in the worklist being empty.  During the
execution of this phase, the residue contains the parse results in
reverse order.  Only at the end of the phase is it reversed so that
the parse results are then in the same order as they appear in the
buffer.

At the beginning of the invalidation phase, parse results are moved
between the prefix and the suffix so that the last parse result of the
prefix entirely precedes the first update operation, and the first
parse result of the suffix does not.

The invalidation phase consists of an \emph{outer} loop and an
\emph{inner} loop.  In the outer loop, each update operation is
processed in increasing order of the line of the operation.  The
invalidation phase terminates when every update operation has been
processed.  The inner loop is executed until one of the following
conditions is met:

\begin{enumerate}
\item Both the worklist and the suffix are empty.
\item The worklist is non-empty, and the position of first parse
  result on the worklist is beyond the line of the update operation
  being processed.
\item The worklist is empty, the suffix is non-empty, and the position
  of first parse result in the suffix is beyond the line of the update
  operation being processed.
\end{enumerate}

In each iteration of the inner loop of the invalidation phase, a
single parse result is processed, namely the first parse result on the
worklist.  If the worklist is empty, the first parse result of the
suffix is moved to the worklist before processing begins.

In each iteration of the inner loop of the invalidation phase, there
are two possible cases:

\begin{enumerate}
\item The parse result to be processed entirely precedes the line of
  the update operation being processed.  In this case, the parse
  result is moved to the residue.
\item The parse result spans the line of the update operation.  Then
  the parse result is removed from the worklist, and its children (if
  any) are pushed on the worklist, preserving their order.
\end{enumerate}

\subsection{Rehabilitation phase}

\subsubsection{Description}

Once the cache has been partially invalidated according to
modifications to the buffer, the buffer must be parsed again so that a
complete valid cache is again obtained.  To avoid having to parse the
entire buffer from beginning to end, we use two crucial ways to speed
up the process:

\begin{enumerate}
\item We do not have to take into account the buffer contents that
  corresponds to the parse results in the prefix of the cache.
  Because of the way the prefix and the suffix were positioned prior
  to the invalidation phase, the parse results in the prefix all
  precede the first modified line of the buffer, so these parse
  results are still valid.
\item When a call to \texttt{read} is made at a particular position in
  the buffer, we first consult the cache.  If the cache contains a
  parse result that was obtained from a previous call to \texttt{read}
  at this position, the entire invocation of the reader is
  short-circuited, so that the existing parse result is returned
  instead, and the stream position is advanced to be positioned at the
  end of that existing parse result.
\end{enumerate}

In order to accomplish the second speedup, we define a new kind of
input stream, using the proposal%
\footnote{See:
  http://www.nhplace.com/kent/CL/Issues/stream-definition-by-user.html
  for a description of the proposal by David Gray.  The proposal did
  not make it into the \commonlisp{} standard, but most modern
  implementations support it.}
 by David Gray for allowing user-defined stream classes in
 \commonlisp{}.  By using such a ``Gray stream'', we avoid having to
 modify the reader while still allowing it to consult the cache before
 starting the normal character-by-character reading process.

Our custom stream uses the \emph{residue} and the \emph{suffix} lists
to guide the reading process.  The reader is invoked repeatedly until
one of the following situations occurs:

\begin{enumerate}
\item The residue and the suffix are both empty.  The prefix then
  contains every valid top-level parse result of the buffer.
\item The residue is empty and the current invocation of the reader
  occurs at the position of the first parse result of the suffix.
  This situation indicates that repeated invocations of the reader
  would return the exact parse results of the suffix.  We can
  therefore stop, knowing that the prefix and the suffix together
  contain every valid top-level parse result of the buffer.
\end{enumerate}

After a top-level invocation of the reader, the current stream
position may have advanced to a point beyond some or all of the parse
results of the residue and possibly also of the suffix.  This
situation occurs when a character that requires nested calls to the
reader is encountered.  These nested calls can then return some of the
cached parse results of the residue and the suffix so that they become
part of a bigger top-level parse result.  For that reason, after each
top-level invocation of the reader, we must discard cached parse
results in the residue and the suffix that precede the end of the
top-level parse result that was returned.

\subsubsection{Example}

To illustrate the rehabilitation phase, let us say that the modified
line $37$ consisted of replacing the variable \texttt{y} by the
binding \texttt{(y 2)}, and that the inserted line is still empty.

The stream position is initially positioned at the end of the last
parse result of the prefix, so that the first left parenthesis of the
line starting the \texttt{let} form is the next character to be read.
The reader calls itself recursively to read the children of the
\texttt{let} form.  When the reader is recursively invoked to read the
first child of the \texttt{let} form, we notice that the symbol
\texttt{let} is in the cache.  Therefore, the cached result is
returned and the stream position is advanced so that it is located
immediately after the symbol \texttt{let}.

When the stream position is located on the left parenthesis of the
binding of the variable \texttt{y}, there is no parse result in the
cache associated with that position.  The reader therefore processes
the input character by character, as it does in its usual mode of
operation.  It is not until the symbol \texttt{g} is processed, that a
cached parse result is again found.

This process continues in that the reader alternately proceeds
character by character, and alternately by leaps according to the
contents of the cache.  When the top-level call to the reader returns,
the original top-level parse result representing the \texttt{let} form
has been created, with the exception of the modified binding of the
variable \texttt{y} and the line-number information modified by the
inserted line.

Once the top-level invocation of read is terminated, the position of
the input stream will be located after the \texttt{let} form.  Cached
parse results in the residue and the suffix preceding this position
are then discarded.  In this case, every parse result in the residue
is discarded, because these parse results are now children of the
newly returned top-level parse result.  Finally, the new parse result
is pushed onto the prefix.

The position of the next top-level invocation of the reader
corresponds to the first position of the first cached parse result of
the suffix.  We therefore know that subsequent invocations of
top-level reads will just return successive parse results of the
suffix, so we can stop, having terminated the rehabilitation phase.
