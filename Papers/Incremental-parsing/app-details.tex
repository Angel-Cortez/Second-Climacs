\section{Cache representation}
\label{app-cache-representation}

\refFig{fig-cache} illustrates the representation of the cache for
parse results.  The buffer contents that corresponds to that cache
contents might for instance be:

\begin{verbatim}
(a
 (b c))
(d)
(e
 f)
(g 
 (h))
\end{verbatim}

The sequence of top-level parse results is split into a \emph{prefix}
and a \emph{suffix}, typically reflecting the current position in the
buffer being edited by the end user.  The suffix contains parse
results in the order they appear in the buffer, whereas the prefix
contains parse results in reverse order, making it easy to move parse
results between the prefix and the suffix.

Depending on the location of the parse result in the cache data
structure, its position may be \emph{absolute} or \emph{relative}.
The prefix contains parse results that precede updates to the buffer.
For that reason, these parse results have absolute positions.  Parse
results in the suffix, on the other hand, follow updates to the
buffer.  In particular, if a line is inserted or deleted, the parse
results in the suffix will have their positions changed.  For that
reason, only the first parse result of the suffix has an absolute
position.  Each of the others has a position relative to its
predecessor.  When a line is inserted or deleted, only the first parse
result of the suffix has to have its position updated.  When a parse
result is moved from the prefix to the suffix, or from the suffix to
the prefix, the positions concerned are updated to maintain this
invariant.

To avoid having to traverse all the descendants of a parse result when
its position changes, we make the position of the first child of some
parse result $P$ relative to that of $P$, and the children, other than
the first, of some parse result $P$, have positions relative to the
previous child in the list.

\begin{figure}
\begin{center}
\inputfig{fig-cache.pdf_t}
\end{center}
\caption{\label{fig-cache}
Representation of the cache.}
\end{figure}

As a result of executing the invalidation phase, a third sequence of
parse results is created.  This sequence is called the \emph{residue},
and it contains valid parse results that were previously children of
some top-level parse result that is no longer valid.  So, for example,
if the line containing the symbol \texttt{f} in the buffer
corresponding to the cache in \refFig{fig-cache} were to be modified,
the result of the invalidation phase would be the cache shown in
\refFig{fig-after-invalidation}.

\begin{figure}
\begin{center}
\inputfig{fig-after-invalidation.pdf_t}
\end{center}
\caption{\label{fig-after-invalidation}
Cache contents after invalidation.}
\end{figure}

As \refFig{fig-after-invalidation} shows, the top-level parse result
corresponding to the expression \texttt{(e f)} has been invalidated,
in addition the child parse result corresponding to the expression
\texttt{f}.  However, the child parse result corresponding to the
expression \texttt{e} is still valid, so it is now in the residue
sequence.  Furthermore, the suffix sequence now contains only the
parse result corresponding to the expression \texttt{(g (h))}.

For the rehabilitation phase, we can imagine that a single character
was inserted after the \texttt{f}, so that the line now reads as
\texttt{fi)}.

At the start of the rehabilitation phase, the position for reading is
set to the end of the last valid top-level parse result in the prefix,
in this case at the end of the line containing the expression
\texttt{(d)}.  When the reader is called, it skips whitespace
characters until it is positioned on the left parenthesis of the line
containing \texttt{(e}.  There is no cache entry, neither in the
residue nor in the suffix, corresponding to this position, so normal
reader operation is executed.  Thus, the reader macro associated with
the left parenthesis is invoked, and the reader is called recursively
on the elements of the list.  When the reader is called with the
position corresponding to the expression \texttt{e}, we find that
there is an entry for that position in the residue, so instead of this
expression being read by normal reader operation, the contents of the
cache is used instead.  As a result, the position in the buffer is set
to the end of the cached parse result, i.e. at the end of the
expression \texttt{e}.  The remaining top-level expression is read
using then normal reader operation resulting in the expression \texttt{(e
  fi)}.  This parse result is added to the prefix resulting in the
cache contents shown in \reffig{fig-after-read}.

\begin{figure}
\begin{center}
\inputfig{fig-after-read.pdf_t}
\end{center}
\caption{\label{fig-after-read}
Cache contents after read.}
\end{figure}

The reader is then invoked again in order to read another top-level
expression.  In this invocation, whitespace characters are first
skipped until the reader is positioned immediately before the
expression \texttt{(g (h))}.  Not only is there a parse result in the
cache corresponding to this position, but that parse result is the
first in the suffix sequence.  We therefore know that all parse
results on in the suffix are still valid, so the we can terminate the
rehabilitation phase.
