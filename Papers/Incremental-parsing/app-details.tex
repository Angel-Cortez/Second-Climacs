\section{Cache representation}
\label{app-cache-representation}

The sequence of top-level parse results is split into a \emph{prefix}
and a \emph{suffix}, typically reflecting the current position in the
buffer being edited by the end user.  The suffix contains parse
results in the order they appear in the buffer, whereas the prefix
contains parse results in reverse order, making it easy to move parse
results between the prefix and the suffix.

Depending on the location of the parse result in the cache data
structure, its position may be \emph{absolute} or \emph{relative}.
The prefix contains parse results that precede updates to the buffer.
For that reason, these parse results have absolute positions.  Parse
results in the suffix, on the other hand, follow updates to the
buffer.  In particular, if a line is inserted or deleted, the parse
results in the suffix will have their positions changed.  For that
reason, only the first parse result of the suffix has an absolute
position.  Each of the others has a position relative to its
predecessor.  When a line is inserted or deleted, only the first parse
result of the suffix has to have its position updated.  When a parse
result is moved from the prefix to the suffix, or from the suffix to
the prefix, the positions concerned are updated to maintain this
invariant.

To avoid having to traverse all the descendants of a parse result when
its position changes, we make the position of the first child of some
parse result $P$ relative to that of $P$, and the children, other than
the first, of some parse result $P$, have positions relative to the
previous child in the list.

\subsection{Invalidation phase}

\subsubsection{Description}

During the invalidation phase, two additional lists of parse results
are maintained, called the \emph{worklist} and the \emph{residue}.  At
the beginning of this phase, these lists are both empty.  At the end
of this phase, any remaining parse results on the worklist are moved
to the residue, resulting in the worklist being empty.  During the
execution of this phase, the residue contains the parse results in
reverse order.  Only at the end of the phase is it reversed so that
the parse results are then in the same order as they appear in the
buffer.

At the beginning of the invalidation phase, parse results are moved
between the prefix and the suffix so that the last parse result of the
prefix entirely precedes the first update operation, and the first
parse result of the suffix does not.

The invalidation phase consists of an \emph{outer} loop and an
\emph{inner} loop.  In the outer loop, each update operation is
processed in increasing order of the line of the operation.  The
invalidation phase terminates when every update operation has been
processed.  The inner loop is executed until one of the following
conditions is met:

\begin{enumerate}
\item Both the worklist and the suffix are empty.
\item The worklist is non-empty, and the position of first parse
  result on the worklist is beyond the line of the update operation
  being processed.
\item The worklist is empty, the suffix is non-empty, and the position
  of first parse result in the suffix is beyond the line of the update
  operation being processed.
\end{enumerate}

In each iteration of the inner loop of the invalidation phase, a
single parse result is processed, namely the first parse result on the
worklist.  If the worklist is empty, the first parse result of the
suffix is moved to the worklist before processing begins.

In each iteration of the inner loop of the invalidation phase, there
are two possible cases:

\begin{enumerate}
\item The parse result to be processed entirely precedes the line of
  the update operation being processed.  In this case, the parse
  result is moved to the residue.
\item The parse result spans the line of the update operation.  Then
  the parse result is removed from the worklist, and its children (if
  any) are pushed on the worklist, preserving their order.
\end{enumerate}
