\section{Reader customization}
\label{app-reader-customization}

In order for it to be possible for the \commonlisp{} \texttt{read}
function to serve as a basis for the incremental parser described in
this paper, it must be adapted in the ways described below.

\subsection{Returning parse results}

In addition to the nested expressions returned by an unmodified
\texttt{read} function, it must also return a nested structure of
\emph{parse results}, i.e. expressions wrapped in standard instances
that also contain information about the location in the source code of
the wrapped expressions.

To accomplish this additional functionality, it is not possible to
create a custom \texttt{read} function that returns parse results
\emph{instead of} expressions, simply because the function must handle
custom reader macros, and those reader macros return expressions, and
not parse results.  Also, it would create unnecessary maintenance work
if all standard reader macros had to be modified in order to return
parse results instead of expressions.

It is also not possible to modify the \texttt{read} function to return
the parse result as a second value, in addition to the normal
expression.  One reason is that we would like for the modified
\texttt{read} function to be compatible with the standard version, and
it is not permitted by the \commonlisp{} standard to return additional
values.

Instead, the modified \texttt{read} function maintains an explicit
stack of parse results in parallel with the expressions that are
normally returned.  This explicit stack is kept as the value of a
special variable that our parser accesses after a call to
\texttt{read}.

\subsection{Returning parse results for comments}

The modified \texttt{read} function must return parse results that
correspond to source code that the standard \texttt{read} function
does not return, such as comments and expressions that are not
selected by a read-time conditional.  We solve this problem by
checking when a reader macro returns no values, and in that case, a
corresponding parse result is pushed onto the explicit stack mentioned
in the previous section.

\subsection{Intercepting symbol creation}

The modified \texttt{read} function must not call \texttt{intern} in
all situations that the ordinary \texttt{read} function would, and it
must not signal an error when a symbol with an explicit package prefix
does not exist.  For that reason, the modified reader calls a generic
function with the characters of a potential token instead.  The
unmodified \texttt{read} function just calls \texttt{intern}, whereas
the custom \texttt{read} function creates a particular parse result
that represents the token, and that can be exploited by the
editor.
