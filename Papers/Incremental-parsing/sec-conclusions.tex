\section{Conclusions and future work}

Parsing \commonlisp{} source code is only the first step in the
analysis of its structure.  In order to determine the role of each
symbol and other information such as indentation, further analysis is
required.  Such analysis requires a \emph{code walker}, because the
role of a symbol may depend on the definitions of macros to which it
is an argument.  Similarly, in order to compute standard
\emph{indentation}, also requires further analysis.  To implement this
code walker, we consider using the first phase of the \cleavir{}
compiler framework%
\footnote{\cleavir{} is currently part of \sicl{}.  See
  https://github.com/robert-strandh/SICL/tree/master/Code/Cleavir.}

We plan to investigate the use of a new implementation of
\emph{first-class global environments}
\cite{Strandh:2015:ELS:Environments}.  This new implementation of the
existing \clos{} protocol would use \emph{incremental differences} to
the \emph{startup environment}%
\footnote{Recall that the startup environment is the environment in
  which the compiler was invoked.}
so as to define a \emph{compilation environment}%
\footnote{Recall that the compilation environment is the environment
  used the compiler for definitions and side effects of the
  compilation process.}
that is different for each top-level form in the editor buffer.  This
technique wold allow us to restart the compiler in an appropriate
environment without having to process the entire buffer from the
beginning.
