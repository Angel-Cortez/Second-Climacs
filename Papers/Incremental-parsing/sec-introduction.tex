\section{Introduction}

Whether autonomous or part of an integrated development environment,
an editor that caters to \commonlisp{} programmers must analyze the
buffer contents in order to help the programmer understand how this
contents would be analyzed when submitted to the \commonlisp{}
compiler or interpreter.  Furthermore, the editor analysis must be
\emph{fast} so that it is up to date shortly after each keystroke
generated by the programmer.  Miller
\cite{Miller:1968:RTM:1476589.1476628} indicates that an upper bound
on the delay between a keystroke and the updated result is around
$0.1$ seconds.

In order to obtain such speed for the analysis, it must be
\emph{incremental}.  A complete analysis of the entire buffer for each
keystroke is generally not feasible, especially for buffers with a
significant amount of code.

Furthermore, the analysis is necessarily \emph{approximate}.  The
reader macro \#. (hash dot) and the macro \texttt{eval-when} allow for
arbitrary computations at read time and at compile time, and these
computations may influence the environment in arbitrary ways that may
invalidate subsequent, or even preceding analyses, making an analysis
that is both precise and incremental impossible in general.

The question, then, is how approximate the analysis has to be, and how
much of it we can allow ourselves to recompute, given the performance
of modern hardware and modern \commonlisp{} implementations.

In this paper, we describe an analysis technique that represents an
improvement compared with the ones used by the most widespread editors
for \commonlisp{} code used today.  The technique is more precise than
existing ones, because it uses the \commonlisp{} \texttt{read}
function, which is a better approximation than the regular-expression
techniques most frequently used.  We show that our analysis is
sufficiently fast because it is done incrementally, and it requires
very little incremental work for most simple editing operations.

The work in this paper is specific to the \commonlisp{} language.
This language has a number of specific features in terms of its
syntax, some of which make it harder to write a parser for it, and
some of which make it easier:

\begin{itemize}
\item The reader algorithm is defined in terms of a non-tokenizing
  recursive descent parser.  This fact makes our task easier, because
  the result of calling \texttt{read} at any location in the source
  code is well defined and yields a unique result.  For other
  languages, the meaning of some sequence of characters may depend on
  what comes much later.
\item The \commonlisp{} reader is \emph{programmable} in that the
  application programmer can define \emph{reader macros} that invoke
  arbitrary code.  This feature makes our task harder, because it
  makes it impossible to establish fixed rules for the meaning of a
  sequence of characters in the buffer.  The technique described in
  this paper can handle such arbitrary syntax extensions.
\item As previously mentioned, arbitrary \commonlisp{} code may be
  invoked as part of a call to \texttt{read}, and that code may modify
  the readtable and/or the global environment.  This possibility makes
  our task harder, and we are only able to address some of the
  problems it creates.
\end{itemize}
