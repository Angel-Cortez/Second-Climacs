\section{Our technique}

Our incremental parser parses the contents of a line-oriented buffer,
specified in the form of a \clos{} protocol
\cite{Strandh:2016:CPE:3005729.3005732}.  The buffer is line-oriented
in two different ways:

\begin{enumerate}
\item The editing operations specified by the protocol define a
  \emph{line} abstraction, in contrast to a buffer of GHU Emacs
  \cite{Finseth:1980:TPTa} which exposes a single sequence containing
  newline characters to indicate line separation.
\item The update protocol works on the granularity of a line.  An
  entire line can be reported as being modified, inserted, or
  deleted.
\end{enumerate}

For the purpose of this paper, the update protocol is the part of the
buffer protocol that we are interested in, because we re-parse the
buffer as a result of the update protocol having been invoked.

In order to parse the buffer contents, we use a custom \texttt{read}
function.  This version of the \texttt{read} function differs from the
standard one in the following ways:

\begin{itemize}
\item Instead of returning S-expressions, it returns a nested
  structure of instances of a standard class named
  \texttt{parse-result}.  These instances contain the corresponding
  S-expression and the start and end position (line, column) in the
  buffer of the parse result.
\item The parse results returned by the reader also include entities
  that would normally not be returned by \texttt{read} such as
  comments and (more generally), results of applying reader macros
  that return no values.
\item Instead of attempting to call \texttt{intern} in order to turn a
  token into a symbol, the custom reader returns an instance of a
  standard class named \texttt{token}.
\end{itemize}

The reader from the \sicl{} project%
\footnote{See: https://github.com/robert-strandh/SICL.}
was slightly modified to allow this kind of customization, thereby
avoid the necessity of maintaining the code for a completely separate
reader.

The update buffer protocol is typically invoked after each keystroke
by the end user, and the modifications to the buffer are typically
very modest, in that usually a single line has been modified.  It
would be wasteful, and too slow for large buffers, to re-parse the
entire buffer character by character, each time the update protocol is
invoked.  For that reason, we keep a \emph{cache} of parse results
returned by the customized reader.

This cache is organized as a sequence%
\footnote{Here, we use the word \emph{sequence} in the meaning of a
  set of items organized consecutively, and not in the more
  restrictive meaning defined by the \commonlisp{} standard.}  of
top-level parse results.  Each top-level parse result contains the
parse results returned by nested calls to the reader.  The sequence of
top-level parse results is split into a \emph{prefix} and a
\emph{suffix}, typically reflecting the current position in the buffer
being edited by the end user.  The suffix contains parse results in
the order they appear in the buffer, whereas the prefix contains parse
results in reverse order, making it easy to move parse results between
the prefix and the suffix.

Depending on the location of the parse result in the cache data
structure, its position may be \emph{absolute} or \emph{relative}.
The prefix contains parse results that precede updates to the buffer.
For that reason, these parse results have absolute positions.  Parse
results in the suffix, on the other hand, follow updates to the
buffer.  In particular, if a line is inserted or deleted, the parse
results in the suffix will have their positions changed.  For that
reason, only the first parse result of the suffix has an absolute
position.  Each of the others has a position relative to its
predecessor.  When a line is inserted or deleted, only the first parse
result of the suffix has to have its position updated.  When a parse
result is moved from the prefix to the suffix, or from the suffix to
the prefix, the positions concerned are updated to maintain this
invariant.

To avoid having to traverse all the descendants of a parse result when
its position changes, we make the position of the first child of some
parse result $P$ relative to that of $P$, and the children, other than
the first, of some parse result $P$, have positions relative to the
previous child in the list.

When the buffer is updated, we try to maintain as many parse results
as possible in the cache.  We can think of a buffer update as
resulting in a sequence of operations, sorted by lines in increasing
order.  There can be three different update operations:

\begin{itemize}
\item Modify.  The line has been modified.
\item Insert.  A new line has been inserted.
\item Delete.  An existing line has been deleted.
\end{itemize}

Updating the cache according to a particular sequence of update
operations consists of two distinct phases:

\begin{enumerate}
\item Invalidation of parse results that span a line that has been
  modified, inserted, or deleted.
\item Rehabilitation of the cache according to the updated buffer
  contents.
\end{enumerate}

During the invalidation phase, two additional lists of parse results
are maintained, called the \emph{worklist} and the \emph{residue}.  At
the beginning of this phase, these lists are both empty.  At the end
of this phase, any remaining parse results on the worklist are moved
to the residue, resulting in the worklist being empty.  During the
execution of this phase, the residue contains the parse results in
reverse order.  Only at the end of the phase is it reversed so that
the parse results are then in the same order as they appear in the
buffer.

At the beginning of the invalidation phase, parse results are moved
between the prefix and the suffix so that the last parse result of the
prefix entirely precedes the first update operation, and the first
parse result of the suffix does not.

The invalidation phase consists of an \emph{outer} loop and an
\emph{inner} loop.  In the outer loop, each update operation is
processed in increasing order of the line of the operation.  The
invalidation phase terminates when every update operations has been
processes.  The inner loop is executed until one of the following
conditions are met:

\begin{iterate}
\item Both the worklist and the suffix are empty.
\item The worklist is non-empty, and the position of first parse
  result on the worklist is beyond the line of the update operation
  being processed.
\item The worklist is empty, the suffix is non-empty, and the position
  of first parse result in the suffix is beyond the line of the update
  operation being processed.
\end{iterate}

In each iteration of the inner loop of the invalidation phase a single
parse result is processed, namely the first parse result on the
worklist.  If the worklist is empty, the first parse result of the
suffix is moved to the worklist before processing begins.

In each iteration of the inner loop of the invalidation phase, there
are two possible cases:

\begin{enumerate}
\item The parse result to be processed entirely precedes the line of
  the update operation being processed.  In this case, the parse
  result is moved to the residue.
\item The parse result spans the line of the update operation.  Then
  the parse result is removed from the worklist, and its children (if
  any) are pushed on the worklist, preserving their order.
\end{enumerate}
