\section{Our technique}

\subsection{Buffer update protocol}
\label{sec-buffer-update-protocol}

Our incremental parser parses the contents of a buffer, specified in
the form of a \clos{} protocol
\cite{Strandh:2016:CPE:3005729.3005732}.  The protocol contains two
sub-protocols:

\begin{enumerate}
\item The \emph{edit protocol} is used whenever items are inserted or
  deleted from the buffer.  An edit operation is typically invoked as
  a result of a keystroke, but an arbitrary number of edit operations
  can happen as a result of a keystroke, for example when a region is
  inserted or deleted, or when a keyboard macro is executed.
\item The \emph{update protocol} is used when the result of an
  arbitrary number of edit operations must be displayed to the user.
  This protocol is typically invoked for each keystroke, but it can be
  invoked less frequently if some view of the buffer is temporarily
  hidden.  Only when the view becomes visible is the update protocol
  invoked.
\end{enumerate}

The buffer protocol is \emph{line-oriented} in two different ways:

\begin{enumerate}
\item The editing operations specified by the protocol define a
  \emph{line} abstraction, in contrast to a buffer of GNU Emacs
  \cite{Finseth:1980:TPTa} which exposes a single sequence containing
  newline characters to indicate line separation.
\item The update protocol works on the granularity of a line.  An
  entire line can be reported as being modified, inserted, or
  deleted.
\end{enumerate}

For the purpose of this paper, we are only interested in the update
protocol, because we re-parse the buffer as a result of the update
protocol having been invoked.  We can think of such an invocation as
resulting in a succession of operations, sorted by lines in increasing
order.  There can be three different update operations:

\begin{itemize}
\item Modify.  The line has been modified.
\item Insert.  A new line has been inserted.
\item Delete.  An existing line has been deleted.
\end{itemize}

In order to parse the buffer contents, we use a custom \texttt{read}
function.  This version of the \texttt{read} function differs from the
standard one in the following ways:

\begin{itemize}
\item Instead of returning S-expressions, it returns a nested
  structure of instances of a standard class named
  \texttt{parse-result}.  These instances contain the corresponding
  S-expression and the start and end position (line, column) in the
  buffer of the parse result.
\item The parse results returned by the reader also include entities
  that would normally not be returned by \texttt{read} such as
  comments and, more generally, results of applying reader macros
  that return no values.
\item Instead of attempting to call \texttt{intern} in order to turn a
  token into a symbol, the custom reader returns an instance of a
  standard class named \texttt{token}.
\end{itemize}

The reader from the \sicl{} project%
\footnote{See: https://github.com/robert-strandh/SICL.}  was slightly
modified to allow this kind of customization, thereby avoiding the
necessity of maintaining the code for a completely separate reader.

No changes to the mechanism for handling reader macros is necessary.
Therefore, we handle custom reader macros as well.  Whenever a reader
macro calls \texttt{read} recursively, a nested parse result is
created in the same way as with the standard reader macros.

The buffer update protocol is typically invoked after each keystroke
by the end user, and the modifications to the buffer are typically
very modest, in that usually a single line has been modified.  It
would be wasteful, and too slow for large buffers, to re-parse the
entire buffer character by character, each time the update protocol is
invoked.  For that reason, we keep a \emph{cache} of parse results
returned by the customized reader.

\subsection{Cache organization}

The cache is organized as a sequence%
\footnote{Here, we use the word \emph{sequence} in the meaning of a
  set of items organized consecutively, and not in the more
  restrictive meaning defined by the \commonlisp{} standard.}  of
top-level parse results.  Each top-level parse result contains the
parse results returned by nested calls to the reader.  Here, we are
not concerened with the details of the representation of the cache.
Such details are crucial in order to obtain acceptable performance,
but they are unimportant for understanding the general technique of
incremental parsing.  Refer to \refapp{app-cache-representation} for an
in-depth description of these details.

When the buffer is updated, we try to maintain as many parse results
as possible in the cache.  Updating the cache according to a
particular succession of update operations consists of two distinct
phases:

\begin{enumerate}
\item Invalidation of parse results that span a line that has been
  modified, inserted, or deleted.
\item Rehabilitation of the cache according to the updated buffer
  contents.
\end{enumerate}

\subsection{Invalidation phase}

\subsubsection{Description}

\subsection{Rehabilitation phase}

\subsubsection{Description}

Once the cache has been partially invalidated according to
modifications to the buffer, the buffer must be parsed again so that a
complete valid cache is again obtained.  To avoid having to parse the
entire buffer from beginning to end, we use two crucial ways to speed
up the process:

\begin{enumerate}
\item We do not have to take into account the buffer contents that
  corresponds to the parse results in the prefix of the cache.
  Because of the way the prefix and the suffix were positioned prior
  to the invalidation phase, the parse results in the prefix all
  precede the first modified line of the buffer, so these parse
  results are still valid.
\item When a call to \texttt{read} is made at a particular position in
  the buffer, we first consult the cache.  If the cache contains a
  parse result that was obtained from a previous call to \texttt{read}
  at this position, the entire invocation of the reader is
  short-circuited, so that the existing parse result is returned
  instead, and the stream position is advanced to be positioned at the
  end of that existing parse result.
\end{enumerate}

In order to accomplish the second speedup, we define a new kind of
input stream, using the proposal%
\footnote{See:
  http://www.nhplace.com/kent/CL/Issues/stream-definition-by-user.html
  for a description of the proposal by David Gray.  The proposal did
  not make it into the \commonlisp{} standard, but most modern
  implementations support it.}
 by David Gray for allowing user-defined stream classes in
 \commonlisp{}.  By using such a ``Gray stream'', we avoid having to
 modify the reader while still allowing it to consult the cache before
 starting the normal character-by-character reading process.

Our custom stream uses the \emph{residue} and the \emph{suffix} lists
to guide the reading process.  The reader is invoked repeatedly until
one of the following situations occurs:

\begin{enumerate}
\item The residue and the suffix are both empty.  The prefix then
  contains every valid top-level parse result of the buffer.
\item The residue is empty and the current invocation of the reader
  occurs at the position of the first parse result of the suffix.
  This situation indicates that repeated invocations of the reader
  would return the exact parse results of the suffix.  We can
  therefore stop, knowing that the prefix and the suffix together
  contain every valid top-level parse result of the buffer.
\end{enumerate}

After a top-level invocation of the reader, the current stream
position may have advanced to a point beyond some or all of the parse
results of the residue and possibly also of the suffix.  This
situation occurs when a character that requires nested calls to the
reader is encountered.  These nested calls can then return some of the
cached parse results of the residue and the suffix so that they become
part of a bigger top-level parse result.  For that reason, after each
top-level invocation of the reader, we must discard cached parse
results in the residue and the suffix that precede the end of the
top-level parse result that was returned.

\subsubsection{Example}

To illustrate the rehabilitation phase, let us say that the modified
line $37$ consisted of replacing the variable \texttt{y} by the
binding \texttt{(y 2)}, and that the inserted line is still empty.

The stream position is initially positioned at the end of the last
parse result of the prefix, so that the first left parenthesis of the
line starting the \texttt{let} form is the next character to be read.
The reader calls itself recursively to read the children of the
\texttt{let} form.  When the reader is recursively invoked to read the
first child of the \texttt{let} form, we notice that the symbol
\texttt{let} is in the cache.  Therefore, the cached result is
returned and the stream position is advanced so that it is located
immediately after the symbol \texttt{let}.

When the stream position is located on the left parenthesis of the
binding of the variable \texttt{y}, there is no parse result in the
cache associated with that position.  The reader therefore processes
the input character by character, as it does in its usual mode of
operation.  It is not until the symbol \texttt{g} is processed, that a
cached parse result is again found.

This process continues in that the reader alternately proceeds
character by character, and alternately by leaps according to the
contents of the cache.  When the top-level call to the reader returns,
the original top-level parse result representing the \texttt{let} form
has been created, with the exception of the modified binding of the
variable \texttt{y} and the line-number information modified by the
inserted line.

Once the top-level invocation of read is terminated, the position of
the input stream will be located after the \texttt{let} form.  Cached
parse results in the residue and the suffix preceding this position
are then discarded.  In this case, every parse result in the residue
is discarded, because these parse results are now children of the
newly returned top-level parse result.  Finally, the new parse result
is pushed onto the prefix.

The position of the next top-level invocation of the reader
corresponds to the first position of the first cached parse result of
the suffix.  We therefore know that subsequent invocations of
top-level reads will just return successive parse results of the
suffix, so we can stop, having terminated the rehabilitation phase.
