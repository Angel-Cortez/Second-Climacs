\section{Our technique}

\subsection{Buffer update protocol}
\label{sec-buffer-update-protocol}

Our incremental parser parses the contents of a buffer, specified in
the form of a \clos{} protocol\cite{Strandh:2016:CPE:3005729.3005732}.
We include a brief summary of that protocol here.

The protocol contains two sub-protocols:

\begin{enumerate}
\item The \emph{edit protocol} is used whenever items are inserted or
  deleted from the buffer.  An edit operation is typically invoked as
  a result of a keystroke, but an arbitrary number of edit operations
  can happen as a result of a keystroke, for example when a region is
  inserted or deleted, or when a keyboard macro is executed.
\item The \emph{update protocol} is used when the result of one or
  more edit operations must be displayed to the user.  This protocol
  is typically invoked for each keystroke, but it can be invoked less
  frequently if some view of the buffer is temporarily hidden.  Only
  when the view becomes visible is the update protocol invoked.
\end{enumerate}

This organization solves several problems with the design of similar
protocols:

\begin{itemize}
\item The edit protocol does not trigger any updates of the views.
  The edit operations simply modify the buffer contents, and marks
  modified lines with a \emph{time stamp}.  Therefore the operations
  of the edit protocol are fast.  As a result, complex operations such
  as inserting or deleting a region, or executing a complicated
  keyboard macro, can be implemented as the repeated invocation of
  simpler operations in the edit protocol.  No special treatment is
  required for such complex operations, which simplifies the overall
  design of these operations.
\item There is no need for the equivalent of \emph{observers} as many
  object-oriented design methods require.  Visible views are
  automatically updated after every keystroke.  Each view contains a
  time stamp corresponding to the previous time it was updated, and
  this time stamp is transmitted to the buffer when the update
  protocol is invoked.
\item Views that are invisible are not updated as a result of a
  keystroke.  Such views are updated if and when they again become
  visible.
\end{itemize}

When a view invokes the update protocol, in addition to transmitting
its time stamp to the buffer, it also transmits four \emph{callback
  functions}.  Conceptually, the view contains some mirror
representation of the lines of the buffer.  Before the update protocol
is invoked, the view sets an index into that representation to zero,
meaning the first line.  As  a result of invoking the update protocol,
the buffer informs the view of changes that happened after the time
indicated by the time stamp by calling these callback functions as
follows:

\begin{itemize}
\item The callback function \emph{skip} indicates to the view that the
  index should be incremented by a number given as argument to the
  function.
\item The callback function \emph{modify} indicates a line that has been
  modified since the last update.  The line is passed as an argument.
  The view must delete lines at the current index until the correct
  line is the one at the index.  It must then take appropriate action
  to reflect the modification, typically by copying the new line
  contents into its own mirror data structure.
\item The callback function \emph{insert} indicates that a line has
  been inserted at the current index since the last update.  Again,
  the line is passed as an argument.  The view updates its mirror data
  structure to reflect the new line.
\item The callback function \emph{sync} is called with a line passed
  as an argument.  The view must delete lines at the current index
  until the correct line is the one at the index.
\end{itemize}

Notice that there is no \emph{delete} callback function.  The buffer
does not hold on to lines that have been deleted, so it is incapable
of supplying this information.  Instead, the \emph{modify} and
\emph{sync} operations provide this information implicitly by
supplying the next line to be operated on.  Any lines preceding it in
the mirror data structure are no longer present in the buffer and
should be deleted by the view.

The buffer protocol is \emph{line-oriented} in two different ways:

\begin{enumerate}
\item The editing operations specified by the protocol define a
  \emph{line} abstraction, in contrast to a buffer of GNU Emacs
  \cite{Finseth:1980:TPTa} which exposes a single sequence containing
  newline characters to indicate line separation.
\item The update protocol works on the granularity of a line.  An
  entire line can be reported as being modified or inserted.
\end{enumerate}

In the implementation of the buffer protocol, a line being edited is
represented as a gap buffer.  Therefore, editing operations are very
fast, even for very long lines.  However, the update protocol works on
the granularity of an entire line.  This granularity is acceptable for
\commonlisp{} code, because lines are typically short.  For other
languages it might be necessary to use a different buffer library.

For the purpose of this paper, we are only interested in the update
protocol, because we re-parse the buffer as a result of the update
protocol having been invoked.  We can think of such an invocation as
resulting in a succession of operations, sorted by lines in increasing
order.  There can be three different update operations:

\begin{itemize}
\item Modify.  The line has been modified.
\item Insert.  A new line has been inserted.
\item Delete.  An existing line has been deleted.
\end{itemize}

Although the presence of a \emph{delete} operation may seem to
contradict the fact that no such operation is possible, it is fairly
trivial to derive this operation from the ones that are actually
supported by the update protocol.  Furthermore, this derived set of
operations simplifies the presentation of our technique in the rest of
the paper.

In order to parse the buffer contents, we use a custom \texttt{read}
function.  This version of the \texttt{read} function differs from the
standard one in the following ways:

\begin{itemize}
\item Instead of returning S-expressions, it returns a nested
  structure of instances of a standard class named
  \texttt{parse-result}.  These instances contain the corresponding
  S-expression and the start and end position (line, column) in the
  buffer of the parse result.
\item The parse results returned by the reader also include entities
  that would normally not be returned by \texttt{read} such as
  comments and, more generally, results of applying reader macros
  that return no values.
\item Instead of attempting to call \texttt{intern} in order to turn a
  token into a symbol, the custom reader returns an instance of a
  standard class named \texttt{token}.
\end{itemize}

The reader from the \sicl{} project%
\footnote{See: https://github.com/robert-strandh/SICL.}  was slightly
modified to allow this kind of customization, thereby avoiding the
necessity of maintaining the code for a completely separate reader.

No changes to the mechanism for handling reader macros is necessary.
Therefore, we handle custom reader macros as well.  Whenever a reader
macro calls \texttt{read} recursively, a nested parse result is
created in the same way as with the standard reader macros.

For a visible view, the buffer update protocol is invoked after each
keystroke generated by the end user, and the number of modifications
to the buffer since the previous invocation is typically very modest,
in that usually a single line has been modified.  It would be
wasteful, and too slow for large buffers, to re-parse the entire
buffer character by character, each time the update protocol is
invoked.  For that reason, we keep a \emph{cache} of parse results
returned by the customized reader.

\subsection{Cache organization}

The cache is organized as a sequence%
\footnote{Here, we use the word \emph{sequence} in the meaning of a
  set of items organized consecutively, and not in the more
  restrictive meaning defined by the \commonlisp{} standard.}  of
top-level parse results.  Each top-level parse result contains the
parse results returned by nested calls to the reader.  Here, we are
not concerened with the details of the representation of the cache.
Such details are crucial in order to obtain acceptable performance,
but they are unimportant for understanding the general technique of
incremental parsing.  Refer to \refapp{app-cache-representation} for an
in-depth description of these details.

When the buffer is updated, we try to maintain as many parse results
as possible in the cache.  Updating the cache according to a
particular succession of update operations consists of two distinct
phases:

\begin{enumerate}
\item Invalidation of parse results that span a line that has been
  modified, inserted, or deleted.
\item Rehabilitation of the cache according to the updated buffer
  contents.
\end{enumerate}

\subsection{Invalidation phase}

As mentioned in \refSec{sec-buffer-update-protocol}, the invocation of
the buffer-update protocol results in a sequence descriptions of
operations that describe how the buffer has changed from the previous
invocation.

As far as the invalidation phase is concerned, there are only minor
variations in how the different types of operations are handled.  In
all cases (line modification, line insertion, line deletion), the
existing parse results that straddle a place that has been altered
must be invalidated.  Notice that when a top-level parse result
straddles such a modification, that parse result is invalidated, but
it is very likely that several of its children do not straddle the
point of modification.  Therefore such children are not invalidated,
and are kept in the cache in case they are needed during the
rehabilitation phase.

In addition to the parse results being invalidated as described in the
previous paragraph, when the operation represents the insertion or the
deletion of a line, remaining valid parse results following the point
of the operation must be modified to reflect the fact that they now
have a new start-line position.

As described in \refApp{app-cache-representation}, we designed the
data structure carefully both so that invalidating parse results as a
result of these operations, and modifying the start-line position of
remaining valid parse results can be done at very little cost.

\subsection{Rehabilitation phase}

Conceptually, the rehabilitation phase consist of parsing the entire
buffer from the beginning by calling \texttt{read} until the end of
the buffer is reached.  However, three crucial design elements avoid
the necessity of a full re-analysis:

\begin{itemize}
\item Top level parse results that precede the first modification to
  the buffer do not have to be re-analyzed, because they must return
  the same result as before any modification.
\item When \texttt{read} is called at a buffer position corresponding
  to a parse result that is in the cache, we can simply return the
  cache entry rather than re-analyzing the buffer contents at that
  point.
\item If a top-level call to \texttt{read} is made beyond the last
  modification to the buffer, and there is a top-level parse result in
  the cache at that point, then every remaining top-level parse result
  in the cache can be re-used without any further analysis required.
\end{itemize}
