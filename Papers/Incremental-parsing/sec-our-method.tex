\section{Our technique}

\subsection{Buffer update protocol}

Our incremental parser parses the contents of a line-oriented buffer,
specified in the form of a \clos{} protocol
\cite{Strandh:2016:CPE:3005729.3005732}.  The buffer is line-oriented
in two different ways:

\begin{enumerate}
\item The editing operations specified by the protocol define a
  \emph{line} abstraction, in contrast to a buffer of GNU Emacs
  \cite{Finseth:1980:TPTa} which exposes a single sequence containing
  newline characters to indicate line separation.
\item The update protocol works on the granularity of a line.  An
  entire line can be reported as being modified, inserted, or
  deleted.
\end{enumerate}

For the purpose of this paper, the update protocol is the part of the
buffer protocol that we are interested in, because we re-parse the
buffer as a result of the update protocol having been invoked.

In order to parse the buffer contents, we use a custom \texttt{read}
function.  This version of the \texttt{read} function differs from the
standard one in the following ways:

\begin{itemize}
\item Instead of returning S-expressions, it returns a nested
  structure of instances of a standard class named
  \texttt{parse-result}.  These instances contain the corresponding
  S-expression and the start and end position (line, column) in the
  buffer of the parse result.
\item The parse results returned by the reader also include entities
  that would normally not be returned by \texttt{read} such as
  comments and (more generally), results of applying reader macros
  that return no values.
\item Instead of attempting to call \texttt{intern} in order to turn a
  token into a symbol, the custom reader returns an instance of a
  standard class named \texttt{token}.
\end{itemize}

The reader from the \sicl{} project%
\footnote{See: https://github.com/robert-strandh/SICL.}  was slightly
modified to allow this kind of customization, thereby avoiding the
necessity of maintaining the code for a completely separate reader.

The buffer update protocol is typically invoked after each keystroke
by the end user, and the modifications to the buffer are typically
very modest, in that usually a single line has been modified.  It
would be wasteful, and too slow for large buffers, to re-parse the
entire buffer character by character, each time the update protocol is
invoked.  For that reason, we keep a \emph{cache} of parse results
returned by the customized reader.

\subsection{Cache organization}

This cache is organized as a sequence%
\footnote{Here, we use the word \emph{sequence} in the meaning of a
  set of items organized consecutively, and not in the more
  restrictive meaning defined by the \commonlisp{} standard.}  of
top-level parse results.  Each top-level parse result contains the
parse results returned by nested calls to the reader.  The sequence of
top-level parse results is split into a \emph{prefix} and a
\emph{suffix}, typically reflecting the current position in the buffer
being edited by the end user.  The suffix contains parse results in
the order they appear in the buffer, whereas the prefix contains parse
results in reverse order, making it easy to move parse results between
the prefix and the suffix.

Depending on the location of the parse result in the cache data
structure, its position may be \emph{absolute} or \emph{relative}.
The prefix contains parse results that precede updates to the buffer.
For that reason, these parse results have absolute positions.  Parse
results in the suffix, on the other hand, follow updates to the
buffer.  In particular, if a line is inserted or deleted, the parse
results in the suffix will have their positions changed.  For that
reason, only the first parse result of the suffix has an absolute
position.  Each of the others has a position relative to its
predecessor.  When a line is inserted or deleted, only the first parse
result of the suffix has to have its position updated.  When a parse
result is moved from the prefix to the suffix, or from the suffix to
the prefix, the positions concerned are updated to maintain this
invariant.

To avoid having to traverse all the descendants of a parse result when
its position changes, we make the position of the first child of some
parse result $P$ relative to that of $P$, and the children, other than
the first, of some parse result $P$, have positions relative to the
previous child in the list.

When the buffer is updated, we try to maintain as many parse results
as possible in the cache.  We can think of a buffer update as
resulting in a sequence of operations, sorted by lines in increasing
order.  There can be three different update operations:

\begin{itemize}
\item Modify.  The line has been modified.
\item Insert.  A new line has been inserted.
\item Delete.  An existing line has been deleted.
\end{itemize}

Updating the cache according to a particular sequence of update
operations consists of two distinct phases:

\begin{enumerate}
\item Invalidation of parse results that span a line that has been
  modified, inserted, or deleted.
\item Rehabilitation of the cache according to the updated buffer
  contents.
\end{enumerate}

\subsection{Invalidation phase}

\subsubsection{Description}

During the invalidation phase, two additional lists of parse results
are maintained, called the \emph{worklist} and the \emph{residue}.  At
the beginning of this phase, these lists are both empty.  At the end
of this phase, any remaining parse results on the worklist are moved
to the residue, resulting in the worklist being empty.  During the
execution of this phase, the residue contains the parse results in
reverse order.  Only at the end of the phase is it reversed so that
the parse results are then in the same order as they appear in the
buffer.

At the beginning of the invalidation phase, parse results are moved
between the prefix and the suffix so that the last parse result of the
prefix entirely precedes the first update operation, and the first
parse result of the suffix does not.

The invalidation phase consists of an \emph{outer} loop and an
\emph{inner} loop.  In the outer loop, each update operation is
processed in increasing order of the line of the operation.  The
invalidation phase terminates when every update operation has been
processes.  The inner loop is executed until one of the following
conditions are met:

\begin{enumerate}
\item Both the worklist and the suffix are empty.
\item The worklist is non-empty, and the position of first parse
  result on the worklist is beyond the line of the update operation
  being processed.
\item The worklist is empty, the suffix is non-empty, and the position
  of first parse result in the suffix is beyond the line of the update
  operation being processed.
\end{enumerate}

In each iteration of the inner loop of the invalidation phase a single
parse result is processed, namely the first parse result on the
worklist.  If the worklist is empty, the first parse result of the
suffix is moved to the worklist before processing begins.

In each iteration of the inner loop of the invalidation phase, there
are two possible cases:

\begin{enumerate}
\item The parse result to be processed entirely precedes the line of
  the update operation being processed.  In this case, the parse
  result is moved to the residue.
\item The parse result spans the line of the update operation.  Then
  the parse result is removed from the worklist, and its children (if
  any) are pushed on the worklist, preserving their order.
\end{enumerate}

\subsubsection{Example}

To illustrate the invalidation phase, consider the following buffer
contents:

{\small\begin{verbatim}
    ...
    34 (f 10)
    35
    36 (let ((x 1)
    37       (y 2))
    38   (g (h x)
    39      (i y)
    40      (j x y)))
    41
    42 (f 20)
    ...
\end{verbatim}}

Each line is preceded by the absolute line number.  If the parse
result starting at line 36 is a member of the prefix or if it is the
first element of the suffix, it would be represented like this:

{\small\begin{verbatim}
    36 04 (let ((x 1) (y 2)) (g (h x) (i y) (j x y)))
       00 01 ((x 1) (y 2))
          00 00 (x 1)
          01 00 (y 2)
       02 02 (g (h x) (i y) (j x y))
          00 00 (h x)
          01 00 (i y)
          02 00 (j x y)
\end{verbatim}}

Since column numbers are uninteresting for our illustration, we
show only line numbers.  Furthermore, we present a list as a table for
a more compact presentation.

Suppose, for example, that the buffer contents in our example was
modified so that line 37 was altered in some way, and a line was
inserted between the lines 39 and 40.  As a result of this update, we
need to represent the following parse results:

{\small\begin{verbatim}
    ...
    34 (f 10)
    35
    36       (x 1)
    37
    38      (h x)
    39      (i y)
    40
    41      (j x y)
    42
    43 (f 20)
    ...
\end{verbatim}}

In other words, we need to obtain the following representation:

{\small\begin{verbatim}
    prefix
    ...
    34 00 (f 10)
    residue
    36 00 (x 1)
    38 00 (h x)
    39 00 (i y)
    41 00 (j x y)
    suffix
    43 00 (f 20)
    ...
\end{verbatim}}

Line 37 has been altered, so our first task is to adjust the prefix
and the suffix so that the prefix contains the last parse result that
is unaffected by the modifications.  This adjustment results in the
following situation:

{\small\begin{verbatim}
    prefix
    ...
    34 00 (f 10)
    residue
    worklist
    suffix
    36 04 (let ((x 1) (y 2)) (g (h x) (i y) (j x y)))
    06 00 (f 20)
    ...
\end{verbatim}}

We now start processing the modification of line 37.  The worklist is
empty, so the first action consists of moving the first parse result
from the suffix to the worklist, resulting in the following situation:

{\small\begin{verbatim}
    prefix
    ...
    34 00 (f 10)
    residue
    worklist
    36 04 (let ((x 1) (y 2)) (g (h x) (i y) (j x y)))
    suffix
    42 00 (f 20)
    ...
\end{verbatim}}

The first parse result of the worklist is affected by the fact that
line 37 has been modified.  We must remove that parse result from the
worklist, and push its children to the worklist.  In doing so, we make
the \texttt{start-line} of the children reflect the absolute line
number.  We obtain the following situation:

{\small\begin{verbatim}
    prefix
    ...
    34 00 (f 10)
    residue
    worklist
    36 01 ((x 1) (y 2))
    38 02 (g (h x) (i y) (j x y))
    suffix
    42 00 (f 20)
    ...
\end{verbatim}}

The first element of the worklist is affected by the modification of
line 37.  We therefore remove it from the worklist, and add its
children to the top of the worklist.  In doing so, we make the
\texttt{start-line} of those children reflect absolute line numbers.
We obtain the following situation:

{\small\begin{verbatim}
    prefix
    ...
    34 00 (f 10)
    residue
    worklist
    36 00 (x 1)
    37 00 (y 2)
    38 02 (g (h x) (i y) (j x y))
    suffix
    42 00 (f 20)
    ...
\end{verbatim}}

The first element of the worklist is unaffected by the modification,
because it precedes the modified line entirely.  We therefore move it
to the residue list.  We now have the following situation:

{\small\begin{verbatim}
    prefix
    ...
    34 00 (f 10)
    residue
    36 00 (x 1)
    worklist
    37 00 (y 2)
    38 02 (g (h x) (i y) (j x y))
    suffix
    42 00 (f 20)
    ...
\end{verbatim}}

The first parse result of the top of the worklist is affected by the
modification.  It has no children, so we pop it off the worklist.  We
obtain the following situation:

{\small\begin{verbatim}
    prefix
    ...
    34 00 (f 10)
    residue
    36 00 (x 1)
    worklist
    38 02 (g (h x) (i y) (j x y))
    suffix
    42 00 (f 20)
    ...
\end{verbatim}}

The modification of line 37 is now entirely processed.  We know this
because the first parse result on the worklist occurs beyond the
modified line in the buffer.  We therefore start processing the line
inserted between the existing lines 39 and 40.  The first item on the
worklist is affected by this insertion.  We therefore remove it from
the worklist and push its children instead.  In doing so, we make the
\texttt{start-line} of those children reflect the absolute line
number.  We obtain the following result:

{\small\begin{verbatim}
    prefix
    ...
    34 00 (f 10)
    residue
    36 00 (x 1)
    worklist
    38 00 (h x)
    39 00 (i y)
    40 00 (j x y)
    suffix
    42 00 (f 20)
    ...
\end{verbatim}}

The first element of the worklist is unaffected by the insertion
because it precedes the inserted line entirely.  We therefore move it
to the residue list.  We now have the following situation:

{\small\begin{verbatim}
    prefix
    ...
    34 00 (f 10)
    residue
    36 00 (x 1)
    38 00 (h x)
    worklist
    39 00 (i y)
    40 00 (j x y)
    suffix
    42 00 (f 20)
    ...
\end{verbatim}}

Once again, the first element of the worklist is unaffected by the
insertion because it precedes the inserted line entirely.  We
therefore move it to the residue list.  We now have the following
situation:

{\small\begin{verbatim}
    prefix
    ...
    34 00 (f 10)
    residue
    36 00 (x 1)
    38 00 (h x)
    39 00 (i y)
    worklist
    40 00 (j x y)
    suffix
    42 00 (f 20)
    ...
\end{verbatim}}

The first element of the worklist is affected by the insertion, in
that it must have its line number incremented.  In fact, every element
of the worklist and also the first element of the suffix must have
their line numbers incremented.  Furthermore, this update finishes the
processing of the inserted line.  We now have the following situation:

{\small\begin{verbatim}
    prefix
    ...
    34 00 (f 10)
    residue
    36 00 (x 1)
    38 00 (h x)
    39 00 (i y)
    worklist
    41 00 (j x y)
    suffix
    43 00 (f 20)
    ...
\end{verbatim}}

With no more buffer modifications to process, we terminate the
procedure by moving remaining parse results from the worklist to the
residue list.  The final situation is shown here:

{\small\begin{verbatim}
    prefix
    ...
    34 00 (f 10)
    residue
    36 00 (x 1)
    38 00 (h x)
    39 00 (i y)
    41 00 (j x y)
    worklist
    suffix
    43 00 (f 20)
    ...
\end{verbatim}}

\subsection{Rehabilitation phase}

Once the cache has been partially invalidated according to
modifications to the buffer, the buffer must be parsed again so that a
complete valid cache is again obtained.  To avoid having to parse the
entire buffer from beginning to end, we use two crucial ways to speed
up the process:

\begin{enumerate}
\item We do not have to take into account the buffer contents that
  corresponds to the parse results in the prefix of the cache.
  Because of the way the prefix and the suffix were positioned prior
  to the invalidation phase, the parse results in the prefix all
  precede the first modified line of the buffer, so these parse
  results are still valid.
\item When a call to \texttt{read} is made at a particular position in
  the buffer, we first consult the cache.  If the cache contains a
  parse result that was obtained from a previous call to \texttt{read}
  at this position, the entire invocation of the reader is
  short-circuited, so that the existing parse result is returned
  instead, and the stream position is advanced to be positioned at the
  end of that existing parse result.
\end{enumerate}

In order to accomplish the second speedup, we define a new kind of
input stream, using the proposal%
\footnote{See:
  http://www.nhplace.com/kent/CL/Issues/stream-definition-by-user.html
  for a description of the proposal by David Gray.  The proposal did
  not make it into the \commonlisp{} standard, but most modern
  implementations support it.}
 by David Gray for allowing user-defined stream classes in
 \commonlisp{}.  By using such a ``Gray stream'', we avoid having to
 modify the reader while still allowing it to consult the cache before
 starting the normal character-by-character reading process.

Our custom stream uses the \emph{residue} and the \emph{suffix} lists
to guide the reading process.  The reader is invoked repeatedly until
one of the following situations occur:

\begin{enumerate}
\item The residue and the suffix are both empty.  The prefix then
  contains every valid top-level parse result of the buffer.
\item The residue is empty and the current invocation of the reader
  occurs at the position of the first parse result of the suffix.
  This situation indicates that repeated invocations of the reader
  would return the exact parse results of the suffix.  We can
  therefore stop, knowing that the prefix and the suffix together
  contain every valid top-level parse result of the buffer.
\end{enumerate}

After a top-level invocation of the reader, the current stream
position may have advanced to a point beyond some or all of the parse
results of the residue and possibly also of the suffix.  This
situation occurs when a character that requires nested calls to the
reader is encountered.  These nested calls can then return some of the
cached parse results of the residue and the suffix so that they become
part of a bigger top-level parse result.  For that reason, after each
top-level invocation of the reader, we must discard cached parse
results in the residue and the suffix that precede the end of the
top-level parse result that was returned.

\subsubsection{Example}

To illustrate the rehabilitation phase, let us say that the modified
line $37$ consisted of replacing the initial value $2$ of the variable
\texttt{y} by $3$ instead, and that the inserted line is still empty.

The stream position is initially positioned at the end of the last
parse result of the prefix, so that the first left parenthesis of the
line starting the \texttt{let} binding is the next character to be
read.  The reader calls itself recursively to read the \texttt{let}
form.  During this process, individual characters are read until the
reader is invoked recursively with the start position of binding of
the variable \texttt{x}.  At this point, the cached parse result is
returned instead, and the position of the stream is advanced so that
it is located after this binding.

This process continues in that the reader alternately proceeds
character by character, and alternately by leaps according to the
contents of the cache.  When the top-level call to the reader returns,
the original top-level parse result representing the \texttt{let} form
has been created, with the exception of the modified initial value and
the line-number information modified by the inserted line.

Once the top-level invocation of read is terminated, the position of
the input stream will be located after the \texttt{let} form.  Cached
parse results in the residue and the suffix preceding this position
are then discarded.  In this case, every parse result in the residue
is discarded, because these parse results are now children of the
newly returned top-level parse result.  Finally, the new parse result
is pushed onto the prefix.

The position of the next top-level invocation of the reader
corresponds to the first position of the first cached parse result of
the suffix.  We therefore know that subsequent invocations of
top-level reads will just return successive parse results of the
suffix, so we can stop, having terminated the rehabilitation phase.
