\section{Performance of our technique}

The performance of our technique can not be stated as a single figure,
nor even as a function of the size of the buffer, simply because it
depends on several factors such as the exact structure of the buffer
contents and the way the user interacts with that contents.

Despite these difficulties, we can give some indications for certain
important special cases.  We ran these tests on a $4$-core Intel Core
processor clocked at $3.3$GHz, running SBCL version 1.3.11.

\subsection{Parsing with an empty cache}

When a buffer is first created, the cache is empty.  The buffer
contents must then be read, character by character, and the cache must
be created from the contents.

We timed this situation with a buffer containing $10000$ lines of
representative \commonlisp{} code.  The total time to parse was around
$1.5$ seconds.  This result deserves some clarifications:

\begin{itemize}
\item It is very unusual to have a file of \commonlisp{} code with
  this many lines.  Most files contain less than $2000$ lines, which
  is only $1/5$ of the one in our test case.
\item This result was obtained from a very preliminary version of our
  parser.  In particular, to read a character, several generic
  functions where called, including the \texttt{stream-read-char}
  function of the Gray streams library, and then several others in
  order to access the character in the buffer.  Further optimizations
  are likely to decrease the time spent to read a single character.
\item This situation will happen only when a buffer is initially read
  into the editor.  Even very significant subsequent changes to the
  contents will still preserve large portions of the cache, so that
  the number of characters actually read will only be a tiny fraction
  of the total number of characters in the buffer.
\item Parsing an entire buffer does not exercise the incremental
  aspect of our parser.  Instead, the execution time is a
  complex function of the exact structure of the code, the performance
  of the reader in various situations, the algorithm for
  generic-function dispatch of the implementation, the cost of
  allocating standard objects, etc.  For all these reasons, a more
  thorough analysis of this case is outside the scope of this paper,
  and the timing is given only to give the reader a rough idea of the
  performance in this initial situation.
\item This particular case can be handled by having the parser process
  the original stream from which the buffer contents was created,
  rather than giving it the buffer protocol wrapped in a stream
  protocol after the buffer has been filled.  That way, the entire
  overhead of the Gray-stream protocol is avoided altogether.
\end{itemize}

\subsection{Parsing after small modifications}

We measured the time to update the cache of a buffer with 1200 lines
of \commonlisp{} code.  We used several variations on the number of
top-level forms and the size of each top-level form.  Three types of
representative modifications were used, namely inserting/deleting a
constituent character, inserting/deleting left parenthesis, and
inserting/deleting a double quote.  All modifications were made at the
very beginning of the file, which is the worst-case scenario for our
technique.

For inserting and deleting a constituent character, we obtained the
results shown in \refTab{t-insert-delete-x}.  For this benchmark, the
performance is independent of the distribution of forms and sub-forms,
and also of the number of characters in a line.  The execution time is
roughly proportional to the number of lines in the buffer.  For that
reason, the form size is given only in number of lines.
The table shows that the parser is indeed very fast for this kind of
incremental modification to the buffer.

\begin{table}[htb]
  \centering{
\begin{tabular}{|r|r|r|}
\hline
forms & form size & time\\
\hline
120 & 10 & 0.14ms\\
80 & 15  & 0.14ms\\
60 & 20  & 0.14ms\\
24 & 100 & 0.23ms\\
36 & 100 & 0.32ms\\
\hline
\end{tabular}
\caption{\label{t-insert-delete-x}
  Inserting and deleting a constituent character.}
}
\end{table}

For inserting and deleting a left parenthesis, we obtained the results
shown in \refTab{t-insert-delete-parenthesis}.  For this benchmark,
the performance is independent of the size of the sub-forms of the
top-level forms.  For that reason, the form size is given only in
number of lines.  As shown in the table, the performance is worse for
many small top-level forms, and then the execution time is roughly
proportional to the number of forms.  When the number of top-level
forms is small, the execution time decreases asymptotically to around
0.5ms.  However, even the slowest case is very fast and has no impact
on the perceived overall performance of the editor.

\begin{table}[htb]
  \centering{
    \begin{tabular}{|r|r|r|}
      \hline
      forms & form size & time\\
      \hline
      120 & 10 & 1.3ms\\
      80 & 15  & 1.0ms\\
      60 & 20  & 0.5ms\\
      40 & 30  & 0.7ms\\
      30 & 40  & 0.6ms\\
      24 & 50  & 0.5ms\\
      12 & 100 & 0.5ms\\
      \hline
    \end{tabular}
    \caption{\label{t-insert-delete-parenthesis}
      Inserting and deleting a left parenthesis.}
  }
\end{table}

Finally, for inserting and deleting a double quote, we obtained the
results shown in \refTab{t-insert-delete-double-quote}.  For this
benchmark, the performance is roughly proportional to the number of
characters in the buffer when the double quote is inserted, and
completely dominated by the execution time of the reader when the
double quote is deleted.  The execution time thus depends not only on
the number of characters in the buffer, but also on how those
characters determine what the reader does.  As shown by the table,
these execution times are borderline acceptable.  In the next section,
we discuss possible ways of improving the performance for this case.

\begin{table}[htb]
  \centering{
    \begin{tabular}{|r|r|r|r|}
      \hline
      forms & form size (lines) & characters per line & time\\
      \hline
      120 & 10 & 1 & 18ms\\
      80 & 15  & 1 & 15ms\\
      60 & 20  & 1 & 17ms\\
      24 & 100 & 1 & 33ms\\
      36 & 100 & 1 & 50ms\\
      120 & 10 & 30 & 70ms\\
      \hline
    \end{tabular}
    \caption{\label{t-insert-delete-double-quote}
      Inserting and deleting a double quote.}
  }
\end{table}
