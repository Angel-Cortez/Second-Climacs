\section{Previous work}

In this section, we present a selection of existing editors, and in
particular, we discuss the technique that each selected editor uses in
order to analyze a text buffer containing \commonlisp{} code.

We do not cover languages other than \commonlisp{}, simply because our
technique crucially depends on ability to analyse the buffer contents
with a non-tokenizing (i.e, based on reading a character at a time)
recursive descent parser.  The \commonlisp{} \texttt{read} function is
defined this way, but most other languages require more sophisticated
parsing techniques for a correct analysis.

\subsection{\emacs{}}
\label{sec-previous-emacs}

GNU Emacs \cite{Finseth:1980:TPTa,CraftOfTextEditiing} is a
general-purpose text editor written partly in \clanguage{} but mostly
in a special-purpose dialect of \lisp{}
\cite{GNUEmacsLispReferenceManual}.

In the editing mode used for writing \commonlisp{} source code,
highlighting is based on string matching, and no attempt is made to
determine the symbols that are present in the current package.  Even
when the current package does not use the \texttt{common-lisp}
package, strings that match \commonlisp{} symbols are highlighted
nevertheless.

In addition, no attempt is made to distinguish between the role of
different occurrences of a symbol.  In \commonlisp{} where a symbol
can simultaneously be used to name a function, a variable, etc., it
would be desirable to present occurrences with different roles in a
different way.

Indentation is also based on string matching, resulting in the text
being indented as \commonlisp{} code even when it is not.
Furthermore, indentation does not take into account the role of a
symbol in the code.  So, for example, if a lexical variable named
(say) \texttt{prog1} is introduced in a \texttt{let} binding and
it is followed by a newline, the following line is indented as if the
symbol \texttt{prog1} were the name of a \commonlisp{} function as
opposed to the name of a lexical variable.

\subsection{\climacs{}}

\climacs%
\footnote{See: https://common-lisp.net/project/climacs/}
is an \emacs{}-like editor written entirely in \commonlisp{}.  It uses
McCLIM \cite{Strandh:2002:ILC:McCLIM}
for its user interface, and specifically, an additional library called
ESA \cite{Strandh:2007:ECL:1622123.1622150}.

The framework for syntax analysis in \climacs{}
\cite{Rhodes.etal:2005} is very general.  The parser for \commonlisp{}
syntax is based on table-driven parsing techniques such as LALR
parsing, except that the parsing table was derived manually.  Like
\emacs{}, \climacs{} does not take the current package into account.
The parser is incremental in that the state of the parser is saved for
certain points in the buffer, so that parsing can be restarted from
such a point without requiring the entire buffer to be parsed from the
beginning.

Unlike \emacs{}, the \climacs{} parser is more accurate when it comes
to the role of symbols in the program code.  In many cases, it is able
to distinguish between a symbol used as the name of a function and the
same symbol used as a lexical variable.

\subsection{Lem}

Lem%
\footnote{See https://github.com/cxxxr/lem.}
is a relatively recent addition to the family of \emacs{} clones.  It
is written in \commonlisp{} and it uses \texttt{curses} to display
buffer contents.

Like \emacs{} \seesec{sec-previous-emacs}, it uses regular expressions
for analyzing \commonlisp{} code, with the same disadvantages in terms
of precision of the analysis.
