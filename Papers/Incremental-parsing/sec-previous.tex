\section{Previous work}

\subsection{Emacs}

Highlighting is based on string matching, and no attempt is made to
determine the symbols that are present in the current package.
Even when the current package does not use the \texttt{common-lisp}
package, strings that match \commonlisp{} symbols are highlighted
nevertheless.

Indentation is also based on string matching, resulting in the text
being indented as \commonlisp{} code even when it is not.
Furthermore, indentation does not take into account the role of a
symbol in the code.  So, for example, if a lexical variable named
(say) \texttt{prog1} is introduced in a \texttt{let} binding and
it is followed by a newline, the following line is indented as if the
symbol \texttt{prog1} were the name of a \commonlisp{} function as
opposed to the name of a lexical variable.
